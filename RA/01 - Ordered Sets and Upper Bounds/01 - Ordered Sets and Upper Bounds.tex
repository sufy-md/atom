\documentclass[a4paper, 14pt]{extarticle}

\usepackage{amsmath}
\usepackage{amssymb}
\usepackage{amsthm}
\usepackage{commath}
\usepackage[margin=0.5in]{geometry}
% \usepackage{lexend}
\usepackage{microtype}
\usepackage{parskip}
\usepackage{tikz}
\usepackage{tikz-cd}
\usepackage{tkz-euclide}
\usepackage{xparse}

\usetikzlibrary{calc,angles,quotes, positioning, shapes.geometric}

\theoremstyle{definition}
\newtheorem{dfn}{Definition}
\newtheorem{clm}{Claim}
\newtheorem{asn}{Assertion}
\newtheorem{thm}{Theorem}
\newtheorem{prb}{Problem}
\newtheorem{ans}{Answer}
\newtheorem{lm}{Lemma}
\newtheorem{rmk}{Remark}
\newtheorem{crl}{Corollary}
\newtheorem{ex}{Exercise}
\newtheorem{xmp}{Example}

\newcommand{\titleheader}[1]{\begin{centering}
\begin{LARGE}
\textbf{#1}
\end{LARGE}\\
\hrulefill

\vspace{-1.25\baselineskip}
\hrulefill
\end{centering}\\\\}

\def\changemargin#1#2{\list{}{\rightmargin#2\leftmargin#1}\item[]}
\let\endchangemargin=\endlist

\NewDocumentEnvironment{smrg}{}{\begin{changemargin}{0.5cm}{0.5cm}}{\end{changemargin}}

\NewDocumentEnvironment{SWP}{m m}
{%
  \vspace{-.9cm}%
  \begin{changemargin}{0.5cm}{0.5cm}%
  \noindent#1~#2
  \par
  \textbf{Proof.} 
}
{%
  \qed
  \end{changemargin}
}

\NewDocumentEnvironment{SNP}{m}
{%
  \vspace{-.9cm}%
  \begin{changemargin}{0.5cm}{0.5cm}%
  \noindent#1
}
{%
  \end{changemargin}
}

\newcommand{\bb}[1]{\mathbb{#1}}
\newcommand{\st}{\space \mid \space}
\newcommand{\union}[1]{\displaystyle\mathop{\cup}\limits_{#1}}
\newcommand{\intersect}[1]{\displaystyle\mathop{\cap}\limits_{#1}}
\newcommand{\paran}[1]{\left ( {#1} \right )}
\newcommand{\contra}{$\rightarrow\!\leftarrow$}
\newcommand{\kvec}[2]{({#1}_1 \dots {#1}_{#2})}
\newcommand{\pf}{\textbf{Proof.} }

\newcommand{\AnswerSection}{
    \newpage
    \section*{Answers to Exercises}
    \textit{The following are brief solutions or hints. You are encouraged to review the exercises before checking the answers.}
}
\begin{document}
\titleheader{Ordered Sets and Upper Bounds}
Our discussion of calculus begins at ordered sets. Calculus is done on the real numbers $\bb R$, and the properties that allow us to do calculus on $\bb R$ are closely linked with its order properties. Although that will not be a part of our main discussion, it is helpful to start there.
\begin{SNP}{\dfn}{Let $S$ be a set. Let $<$ be a binary relation on $S$ satisfying
\begin{itemize}
	\item $<$ is antireflexive, i.e for any $a \in S$, $a \not < a$
	\item $<$ is transitive, i.e if $a < b$ and $b < c$ then $a < c$
	\item $<$ is total, i.e if $a \neq b$ then either $a < b$ or $b < a$
\end{itemize}
Under these circumstances, we say $S$ is \emph{totally ordered} and $<$ is an \emph{ordering} on $S$.}
\end{SNP}

All our familiar sets $\bb N, \bb Z, \bb Q, \bb R$ are ordered, while a set like $\bb C$ does not admit a natural ordering. It is, however, possible to define one. Consider two complex numbers $z_1 := x_1 + iy_1$ and $z_2 := x_2 + iy_2$. If $x_1 < x_2$ then declare $z_1 < z_2$, and vice versa. If at all $x_1 = x_2$, then do the same process with $y_1$ and $y_2$. If they are also equal, then $z_1$ and $z_2$ were already equal. Verify that this is a total ordering on $\bb C$.
\begin{SNP}{\ex}{Show that if $a < b$, then $b \not < a$. (Hint. Suppose for some $a$ and $b$, both $a < b$ and $b < a$ happen. Use transitivity. Is this allowed?)}
\end{SNP}
Ordered sets are more or less intuitive. We just define a notion of being big or small on a set of objects.

Consider some currency notes. $1, 2, 5, 10$ etc., except $2000$. We regard the set of currency notes $S = \{1, 2, 5, 10, 20, 50, 100, 200, 500\}$ as a subset of $\bb N$, and $2000$ as an element of $\bb N$.

Observe that for every $s \in S$, $s < 2000$ holds. In this case we say that $2000$ is an \emph{upper bound} of $S$. In fact, $501, 502, 503 \dots$ are all upper bounds of $S$. A natural question at this stage is that ``is $500$ an upper bound of $S?$''. The answer is yes, and we will understand why shortly.

For now, familiarise yourself with this definition
\begin{SNP}{\dfn}{Let $S$ be a totally ordered set with ordering $<$, and let $E \subset S$ be a \textbf{nonempty} subset of $S$. Let $s \in S$ be such that for every $x \in E$, $x \leq s$. Then we say $s$ is an \emph{upper bound} of $E$. We also say $E$ is \emph{bounded above} by $s$.}
\end{SNP}

Note the use of $\leq$ in the definition. This is what allows $500$ to be an upper bound of the set of currency notes except $2000$. Why we allow this is something we shall explore later.

\AnswerSection
\ans If $a < b$ and $b < a$, then by transitivity $a < a$, which cannot happen.

\end{document}