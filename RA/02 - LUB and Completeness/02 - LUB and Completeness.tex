\documentclass[a4paper, 14pt]{extarticle}

\usepackage{amsmath}
\usepackage{amssymb}
\usepackage{amsthm}
\usepackage{commath}
\usepackage[margin=0.5in]{geometry}
% \usepackage{lexend}
\usepackage{microtype}
\usepackage{parskip}
\usepackage{tikz}
\usepackage{tikz-cd}
\usepackage{tkz-euclide}
\usepackage{xparse}

\usetikzlibrary{calc,angles,quotes, positioning, shapes.geometric}

\theoremstyle{definition}
\newtheorem{dfn}{Definition}
\newtheorem{clm}{Claim}
\newtheorem{asn}{Assertion}
\newtheorem{thm}{Theorem}
\newtheorem{prb}{Problem}
\newtheorem{ans}{Answer}
\newtheorem{lm}{Lemma}
\newtheorem{rmk}{Remark}
\newtheorem{crl}{Corollary}
\newtheorem{ex}{Exercise}
\newtheorem{xmp}{Example}

\newcommand{\titleheader}[1]{\begin{centering}
\begin{LARGE}
\textbf{#1}
\end{LARGE}\\
\hrulefill

\vspace{-1.25\baselineskip}
\hrulefill
\end{centering}\\\\}

\def\changemargin#1#2{\list{}{\rightmargin#2\leftmargin#1}\item[]}
\let\endchangemargin=\endlist

\NewDocumentEnvironment{smrg}{}{\begin{changemargin}{0.5cm}{0.5cm}}{\end{changemargin}}

\NewDocumentEnvironment{SWP}{m m}
{%
  \vspace{-.9cm}%
  \begin{changemargin}{0.5cm}{0.5cm}%
  \noindent#1~#2
  \par
  \textbf{Proof.} 
}
{%
  \qed
  \end{changemargin}
}

\NewDocumentEnvironment{SNP}{m}
{%
  \vspace{-.9cm}%
  \begin{changemargin}{0.5cm}{0.5cm}%
  \noindent#1
}
{%
  \end{changemargin}
}

\newcommand{\bb}[1]{\mathbb{#1}}
\newcommand{\st}{\space \mid \space}
\newcommand{\union}[1]{\displaystyle\mathop{\cup}\limits_{#1}}
\newcommand{\intersect}[1]{\displaystyle\mathop{\cap}\limits_{#1}}
\newcommand{\paran}[1]{\left ( {#1} \right )}
\newcommand{\contra}{$\rightarrow\!\leftarrow$}
\newcommand{\kvec}[2]{({#1}_1 \dots {#1}_{#2})}
\newcommand{\pf}{\textbf{Proof.} }

\newcommand{\AnswerSection}{
    \newpage
    \section*{Answers to Exercises}
    \textit{The following are brief solutions or hints. You are encouraged to review the exercises before checking the answers.}
}
\begin{document}
\titleheader{Least Upper Bounds and Completeness}
\textbf{Prereqs} RA-01

Consider the set $[0, 1] \subset \bb R$ and answer these questions
\begin{itemize}
	\item Is this set bounded above?
	\item Is $1$ an upper bound?
	\item Is anything greater than $1$ an upper bound?
	\item Is anything smaller than $1$ an upper bound?
\end{itemize}

Answers
\begin{itemize}
	\item Yes
	\item Yes, $1$ is an upper bound as everything in the subset is $\leq 1$
	\item Yes, in fact every number $\geq 1$ is an upper bound. Let $y$ be such that $1 \leq y$. Let $x \in [0, 1]$. Since $1$ is an upper bound we get $x \leq 1$. Chaining with $1 \leq y$ we get $x \leq y$ so that $y$ is an upper bound.
	\item No. In fact, nothing smaller than $1$ is an upper bound. Let $x < 1$. But $1 \in S$. Thus the definition that $x$ is an upper bound of $S$ if $x \geq s$ for every element $s$ of $S$ fails at $s = 1$.
\end{itemize}

If we replace $S$ to be the set $[0, 1)$ instead, you'll find the answers remain the same, just that the reasoning for the last one has to be slightly different, because $1$ is no longer in $S$. Now we'll say let $x < 1$. If at all $x < 0$ then $x$ clearly cannot be an upper bound. If instead $0 \leq x < 1$, then the number $\frac{1 + x}{2}$ is larger than $x$ and lies in $[0, 1)$. Thus $x$ is not an upper bound for any $x < 1$.

$1$ here has a special property - that no number smaller than it is an upper bound of the set $S$. We'll look at this in more detail with another example.

Revisit the currency notes example, $S = \{1, 2, 5, 10, 20, 50, 100, 200, 500\}$. Answer the questions
\begin{itemize}
	\item Is $S$ bounded above?
	\item Is $500$ an upper bound?
	\item Is anything greater than $500$ an upper bound?
	\item Is anything smaller than $500$ an upper bound? 
\end{itemize}
Answers
\begin{itemize}
	\item Yes
	\item Yes
	\item Yes, everything greater than $500$ is
	\item No, nothing smaller than $500$ is
\end{itemize}
$500$ also appears to enjoy this peculiar property.

There is nothing particular about $1$ and $500$ here. These numbers exist because of an interesting property of the underlying sets ($\bb N$ and $\bb R$) themselves. We took a nonempty bounded set, and produced a number $\alpha$ such that nothing smaller than $\alpha$ is an upper bound. It makes sense, then, to call these numbers the \emph{least upper bound} of the respective subsets $[0, 1]$ and the currency denominations. 

\begin{SNP}{\dfn}{Let $E \subset S$ be nonempty and bounded above by some $s$. A number $\alpha \in S$ is called a \emph{least upper bound} of $E$ if the following conditions hold
\begin{itemize}
	\item $\alpha$ is an upper bound of $E$
	\item for every $\beta < \alpha$, $\beta$ is NOT an upper bound of $E$
\end{itemize}}
\end{SNP}

Finally we quantify the property $\bb N$ and $\bb R$ enjoy

\begin{SNP}
{\dfn}{Let $<$ be an ordering on $S$. Say $S$ is \emph{complete} if $S$ satisfies the following property:

For \emph{any} nonempty bounded subset $E \subset S$, there is some $x \in S$ such that $x$ is the least upper bound of $E$}
\end{SNP}

What this property says, essentially, that we can \emph{always} find a least upper bound for a subset $E$ so long as it is nonempty and bounded. Try this yourself. Take bounded subsets of $\bb N$ and $\bb R$ and find their least upper bounds. I have demonstrated three examples here already.

But why do we specify complete sets? Isn't every ordered set complete? As we'll soon find out, that is not the case.

\begin{SNP}
{\ex}{Show that if $\alpha$ and $\beta$ are two least upper bounds of a set, they are equal. (Hint. If $x$ is an upper bound and $y$ is a least upper bound, can you comment on whether $y \leq x$ holds?)}
\end{SNP}

\AnswerSection
\ans Let $\alpha$ and $\beta$ be two least upper bounds of a set. Since $\alpha$ is a least upper bound and $\beta$ is an upper bound, it holds that $\alpha \leq \beta$. Similarly it holds that $\beta \leq \alpha$. If $\alpha \neq \beta$ then we get $\alpha < \beta$ and $\beta < \alpha$ simultaneously which cannot happen. Thus $\alpha = \beta$.

\end{document}