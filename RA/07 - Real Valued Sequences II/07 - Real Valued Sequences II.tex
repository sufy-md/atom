\documentclass[a4paper, 14pt]{extarticle}

\usepackage{amsmath}
\usepackage{amssymb}
\usepackage{amsthm}
\usepackage{commath}
\usepackage[margin=0.5in]{geometry}
% \usepackage{lexend}
\usepackage{microtype}
\usepackage{parskip}
\usepackage{tikz}
\usepackage{tikz-cd}
\usepackage{tkz-euclide}
\usepackage{xparse}

\usetikzlibrary{calc,angles,quotes, positioning, shapes.geometric}

\theoremstyle{definition}
\newtheorem{dfn}{Definition}
\newtheorem{clm}{Claim}
\newtheorem{asn}{Assertion}
\newtheorem{thm}{Theorem}
\newtheorem{prb}{Problem}
\newtheorem{ans}{Answer}
\newtheorem{lm}{Lemma}
\newtheorem{rmk}{Remark}
\newtheorem{crl}{Corollary}
\newtheorem{ex}{Exercise}
\newtheorem{xmp}{Example}

\newcommand{\titleheader}[1]{\begin{centering}
\begin{LARGE}
\textbf{#1}
\end{LARGE}\\
\hrulefill

\vspace{-1.25\baselineskip}
\hrulefill
\end{centering}\\\\}

\def\changemargin#1#2{\list{}{\rightmargin#2\leftmargin#1}\item[]}
\let\endchangemargin=\endlist

\NewDocumentEnvironment{smrg}{}{\begin{changemargin}{0.5cm}{0.5cm}}{\end{changemargin}}

\NewDocumentEnvironment{SWP}{m m}
{%
  \vspace{-.9cm}%
  \begin{changemargin}{0.5cm}{0.5cm}%
  \noindent#1~#2
  \par
  \textbf{Proof.} 
}
{%
  \qed
  \end{changemargin}
}

\NewDocumentEnvironment{SNP}{m}
{%
  \vspace{-.9cm}%
  \begin{changemargin}{0.5cm}{0.5cm}%
  \noindent#1
}
{%
  \end{changemargin}
}

\newcommand{\bb}[1]{\mathbb{#1}}
\newcommand{\st}{\space \mid \space}
\newcommand{\union}[1]{\displaystyle\mathop{\cup}\limits_{#1}}
\newcommand{\intersect}[1]{\displaystyle\mathop{\cap}\limits_{#1}}
\newcommand{\paran}[1]{\left ( {#1} \right )}
\newcommand{\contra}{$\rightarrow\!\leftarrow$}
\newcommand{\kvec}[2]{({#1}_1 \dots {#1}_{#2})}
\newcommand{\pf}{\textbf{Proof.} }

\newcommand{\AnswerSection}{
    \newpage
    \section*{Answers to Exercises}
    \textit{The following are brief solutions or hints. You are encouraged to review the exercises before checking the answers.}
}
\begin{document}
\titleheader{Real Valued Sequences II}
\textbf{Prereqs} RA-06, INT-02

In INT-02 we establish a definition of convergence of a sequence which I reiterate here.

\begin{SNP}{\dfn}
{
	A sequence $a_n$ \emph{converges to} $L$ (written $a_n \rightarrow L$) if given any $\epsilon > 0$, there is some $N \in \bb N$ such that $\abs{L - a_k} < \epsilon$ for $k = N + 1, N + 2, N + 3 \dots$
}

\end{SNP}

We will now be tasked with some necessary and sufficient conditions on when a sequence converges.

\begin{SNP}{\dfn}{
	Say a sequence $\{a_n\}_n$ is \emph{bounded} if there is some $M > 0$ such that $\abs{a_n} < M$ for every $n$
}
\end{SNP}
\begin{SWP}{\thm
}{Every convergent sequence is bounded.
}
Let $a_n \rightarrow L$. We want to show there is some fixed $M$ such that $\abs{a_n} < M$ for every $n$. Since $a_n \rightarrow L$, by definition for $\epsilon = 1$, there is some $N$ such that all of $a_{N + 1}, a_{N + 2}, a_{N + 3} \dots$ lie within a distance of $1$ from $L$. Thus, the maximum value of $\abs{a_{N + k}}$ is $\abs L + 1$.

So we have dealt with all the terms $a_{N + k}$, and what remains are the terms $\{a_1, a_2, a_3 \dots a_N\}$. But this is a finite set. Thus if we let
$$
M = \max \{ \abs{a_1}, \abs{a_2}, \abs{a_3} \dots \abs{a_N}, \abs L + 1 \}
$$
Then indeed, $\abs{a_n} < M$ for every $n$.
\end{SWP}
\begin{smrg}
\begin{center}
\begin{tikzpicture}[x=4cm]
  % Draw number line
  \draw[->] (-1,0) -- (3,0) node[right] {$x$};

  % Mark point at 1
  \draw (1,0) node[below=4pt] {$L$} -- (1,0.1);

  % Define epsilon
  \def\eps{0.5}

  % Open interval around 1
  \draw[blue, thick] (0.5, 0) -- (1.5, 0);
  \filldraw[fill=white, draw=black, thick] (1 - \eps, 0) circle (2pt);
  \filldraw[fill=white, draw=black, thick] (1 + \eps, 0) circle (2pt);

  % Optional labels
  \node[below=4pt] at (0.5, 0) {$L - 1$};
  \node[below=4pt] at (1.5, 0) {$L + 1$};
\end{tikzpicture}\\
All $a_{N + k}$ lie within the blue region and are therefore bounded \\
$a_1 \dots a_N$ are finitely many terms and are therefore bounded \\
Hence the entire sequence is bounded
\end{center}
\end{smrg}
In general a bounded sequence does not converge, take for example $a_n = (-1)^n$. However, with one extra condition, we get a sufficient condition for convergence.
\begin{SNP}{\dfn}{
	A sequence is \emph{monotone increasing} if $a_n \leq a_{n + 1}$ for every $n$. A sequence is \emph{monotone decreasing} if $a_n \geq a_{n + 1}$ for every $n$. A sequence is called \emph{monotone} if it is either monotone decreasing or monotone increasing.
}
\end{SNP}
\begin{SWP}{\thm}{(Monotone Convergence) A bounded monotone sequence converges.}{
	WLG we will assume $a_n$ is a monotone increasing sequence, the proof for decreasing sequences is similar. Consider the set $\{a_n \st n \in \bb N\}$. The set is both nonempty and bounded, therefore it has a least upper bound, say $\alpha$.

	Let $\epsilon > 0$ be given. Since $\alpha$ is the least upper bound, $\alpha - \epsilon$ can not be an upper bound and therefore, for some $n$, we have
	$$
	\alpha - \epsilon < a_n \leq \alpha
	$$
	Now use that $a_n$ is monotone and $\alpha$ is an upper bound to observe that
	$$
	\alpha - \epsilon < a_n \leq a_{n + 1} \leq a_{n + 2} \leq a_{n + 3} \leq \dots \leq \alpha
	$$
	That is, for every $n' \geq n$ we have $\abs{\alpha - a_{n'}} < \epsilon$ and thus $a_n \rightarrow \alpha$
}
\end{SWP}
\begin{SNP}{\dfn}{
	A sequence $a_n$ is called a \emph{Cauchy sequence} if for every $\epsilon > 0$, there is some $N \in \bb N$, such thaat whenever $m, n \geq N$ then $\abs{a_m - a_n} < \epsilon$
}
\end{SNP}
In other words, a sequence is Cauchy if the points get arbitrarily close to \emph{each other}, rather than just some fixed $L$ as is in convergence. One fact is immediately clear.

\begin{SWP}{\thm}{
	Every convergent sequence is Cauchy
}
Let $a_n \rightarrow L$, and let $\epsilon > 0$. Then, for some $N$, $n \geq N \implies \abs{a_n - L} < \dfrac{\epsilon}{2}$ by definition of convergence. If $m, n$ are greater than $N$, then
\begin{align*}
	\abs{a_m - a_n} &= \abs{a_m - L - (a_n - L)} \\
					&\leq \abs{a_m - L} + \abs{a_n - L} \\
					&\leq \dfrac \epsilon 2 + \dfrac \epsilon 2 \\
					&= \epsilon
\end{align*}
Therefore $a_n$ is Cauchy.
\end{SWP}
It is a subjective matter whether or not you find the following fact surprising, which (for now) I present without proof.
\begin{SNP}{\thm}{
	Every Cauchy sequence converges in $\bb R$
}
\end{SNP}
The proof uses the completeness of $\bb R$ and thus cannot be recreated for $\bb Q$. In fact, we know a Cauchy sequence in $\bb Q$ that does not converge. Consider
$$
3, 3.1, 3.14, 3.141, 3.1415, 3.14159, 3.141592, 3.1415926 \dots
$$
where $a_n$ is given by truncating $\pi$ to $n - 1$ digits after the decimal. We know the sequence converges in $\bb R$, therefore it is Cauchy in $\bb R$, but all its terms belong to $\bb Q$ so it is also Cauchy in $\bb Q$; however it does not converge in $\bb Q$ since $\pi$ is not in $\bb Q$.
\end{document}