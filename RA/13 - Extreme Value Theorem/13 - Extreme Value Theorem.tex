\documentclass[a4paper, 14pt]{extarticle}

\usepackage{amsmath}
\usepackage{amssymb}
\usepackage{amsthm}
\usepackage{commath}
\usepackage[margin=0.5in]{geometry}
% \usepackage{lexend}
\usepackage{microtype}
\usepackage{parskip}
\usepackage{tikz}
\usepackage{tikz-cd}
\usepackage{tkz-euclide}
\usepackage{xparse}

\usetikzlibrary{calc,angles,quotes, positioning, shapes.geometric}

\theoremstyle{definition}
\newtheorem{dfn}{Definition}
\newtheorem{clm}{Claim}
\newtheorem{asn}{Assertion}
\newtheorem{thm}{Theorem}
\newtheorem{prb}{Problem}
\newtheorem{ans}{Answer}
\newtheorem{lm}{Lemma}
\newtheorem{rmk}{Remark}
\newtheorem{crl}{Corollary}
\newtheorem{ex}{Exercise}
\newtheorem{xmp}{Example}

\newcommand{\titleheader}[1]{\begin{centering}
\begin{LARGE}
\textbf{#1}
\end{LARGE}\\
\hrulefill

\vspace{-1.25\baselineskip}
\hrulefill
\end{centering}\\\\}

\def\changemargin#1#2{\list{}{\rightmargin#2\leftmargin#1}\item[]}
\let\endchangemargin=\endlist

\NewDocumentEnvironment{smrg}{}{\begin{changemargin}{0.5cm}{0.5cm}}{\end{changemargin}}

\NewDocumentEnvironment{SWP}{m m}
{%
  \vspace{-.9cm}%
  \begin{changemargin}{0.5cm}{0.5cm}%
  \noindent#1~#2
  \par
  \textbf{Proof.} 
}
{%
  \qed
  \end{changemargin}
}

\NewDocumentEnvironment{SNP}{m}
{%
  \vspace{-.9cm}%
  \begin{changemargin}{0.5cm}{0.5cm}%
  \noindent#1
}
{%
  \end{changemargin}
}

\newcommand{\bb}[1]{\mathbb{#1}}
\newcommand{\st}{\space \mid \space}
\newcommand{\union}[1]{\displaystyle\mathop{\cup}\limits_{#1}}
\newcommand{\intersect}[1]{\displaystyle\mathop{\cap}\limits_{#1}}
\newcommand{\paran}[1]{\left ( {#1} \right )}
\newcommand{\contra}{$\rightarrow\!\leftarrow$}
\newcommand{\kvec}[2]{({#1}_1 \dots {#1}_{#2})}
\newcommand{\pf}{\textbf{Proof.} }

\newcommand{\AnswerSection}{
    \newpage
    \section*{Answers to Exercises}
    \textit{The following are brief solutions or hints. You are encouraged to review the exercises before checking the answers.}
}
\begin{document}
\titleheader{Extreme Value Theorem}
\textbf{Prereqs} RA-08, RA-11

Let $f:[a, b] \rightarrow \bb R$ be continuous in the pen and paper sense. So, there are two points $(a, f(a))$ and $(b, f(b))$ between which you draw a curve with a pen without lifting it. Is it necessary for the curve to be bounded?

\begin{center}
\begin{tikzpicture}[x=1.5cm, y=2cm]

  % Axes
  \draw[->] (-0.4,0) -- (7,0) node[right] {$x$};
  \draw[->] (0,-0.3) -- (0,3) node[above] {$y$};

  % Define natural log values (approximated)
  \def\lnTwo{0.693}
  \def\lnSix{1.792}

  % Plot ln(x) from x = 2 to x = 6
  \draw[domain=2:6, smooth, variable=\x, blue, thick] 
    plot ({\x}, {ln(\x)});

  % Dashed horizontal lines
  \draw[dashed] (0, \lnTwo) -- (6.5, \lnTwo);
  \draw[dashed] (0, \lnSix) -- (6.5, \lnSix);

  % Vertical dotted lines to x-axis
  \draw[dotted] (2,0) -- (2, {ln(2)});
  \draw[dotted] (6,0) -- (6, {ln(6)});

  % Dots on the curve
  \filldraw[black] (2, {ln(2)}) circle (1.5pt);
  \filldraw[black] (6, {ln(6)}) circle (1.5pt);

  % Labels
  \node[below] at (2, 0) {$a$};
  \node[below] at (6, 0) {$b$};
  \node[left] at (0, \lnTwo) {$f(a)$};
  \node[left] at (0, \lnSix) {$f(b)$};
  % \node[above right] at (4.5, {ln(4.5)}) {$y = \ln x$};

\end{tikzpicture}\\
A continuous curve between two points stays bounded
\end{center}

Intuitively, this makes sense. If we go too far up with the pen, we will need to come back down to match $y = f(b)$ and the function cannot grow unbounded. Let's quantify this formally.

\begin{SWP}{\thm}{(Boundedness) Let $f: [a, b] \rightarrow \bb R$ be continuous. The set $$S := \{f(x) \st x \in [a, b]\}$$is bounded.}Suppose for contradiction that $S$ is not bounded. Then, for every $n$, there is some $y_n \in S$ such that $\abs{y_n} > n$. But every $y$ in $S$ is of the form $f(x)$, thus we get a sequence $\{x_n\}$ with elements in $[a, b]$ such that
$$
\abs{f(x_n)} > n
$$
Note that $\{x_n\}$ is a bounded sequence and therefore has a convergent subsequence, say $x_{n_k} \rightarrow x_0$. But $f$ is continuous, so $f(x_{n_k}) \rightarrow f(x_0)$. Since every convergent sequence is bounded, we get that $\{f(x_{n_k})\}$ is bounded. But $\abs{f(x_{n_k})} > n_k \geq k$ for every $k$ by the assumption that $S$ is unbounded.

Contradiction. Our assumption that $S$ is unbounded fails.
\end{SWP}

\begin{SNP}{\ex}Show with an example that if $[a, b]$ is replaced with $(a, b)$, the theorem doesn't hold. Why can we not just mimic the proof?
\end{SNP}

Certainly $f(a) \in S$. So we can even say $S$ has a least upper bound. We've seen this pattern emerge a few times -- least upper bounds can be approached with sequences, and continuous functions preserve convergence. With this intuition, we state the following result.

\begin{SNP}{\thm}{(Extreme Value) Let $f: [a, b] \rightarrow \bb R$ be continuous. Then, there is an $x_0 \in [a, b]$ such that $f(x_0) \geq f(x)$ for each $x \in [a, b]$}
\end{SNP}
Or in other words, $f$ attains a maximum on $[a, b]$. Note that this is stronger than just being bounded. For example, $f(x) = x$ is bounded by $5$ on $(3, 5)$, in fact it is the least upper bound, but $f$ does not obtain a maximum value on $(3, 5)$.

\begin{smrg}
\textbf{Proof.} Let $\alpha$ be the least upper bound of $S$, then there is a sequence $y_n \rightarrow \alpha$ with $y_n \in S$. But each $y$ is $f(x)$ for some $x$, and hence we get a sequence $\{x_n\}$ such that
$$
f(x_n) \rightarrow \alpha
$$
We need to show that $\alpha = f(x_0)$ for some $x_0$. A sufficient condition for this is $x_n \rightarrow x_0$ because then we can directly use continuity in the above equation. But we don't know if $\{x_n\}$ converges or not.

However, $\{x_n\}$ is bounded, and thus has a convergent subsequence $\{x_{n_k}\}$ converging to some $x_0$. By continuity, we get $f(x_{n_k}) \rightarrow f(x_0)$.

But $f(x_{n_k})$ is a subsequence of $f(x_n)$. Since subsequences converge to the same limit as the original sequence, we get $f(x_{n_k}) \rightarrow \alpha$.

By uniqueness of limit of a sequence, $f(x_0) = \alpha$.
\end{smrg}
\begin{SNP}{\rmk}{The same proof can be recreated to show $f$ attains a minimum.}
\end{SNP}
Once again I've used some results that a standard course should cover, but if not, here they are as exercises.
\begin{SNP}{\ex}Let $x_n \rightarrow L$ and $x_{n_k}$ be any subsequence. Then show that $x_{n_k} \rightarrow L$.
\end{SNP}
\begin{SNP}{\ex}Show that limit of a convergent sequence is unique.
\end{SNP}

There's a subtlety in the proof that's easy to miss. Take a sequence that lies entirely inside $[a, b]$. This is bounded, and thus by Bolzano-Weierstrass, has a convergent subsequence.

What's easy to miss is that the limit of the subsequence also lies inside $[a, b]$. We omitted this detail in the statement and proof provided in RA-08, but is \emph{critical} in the proof of Extreme Value Theorem. For example, $a_n = \frac 1 n$ is a bounded sequence lying entirely inside $(0, 1)$ but does \textbf{NOT} have a limit point in $(0, 1)$. Indeed, this has to do with the fact that $[0, 1]$ is compact while $(0, 1)$ is not (anyone not familiar with Topology can ignore this remark).

In any case, the proof of Bolzano-Weierstrass from Nested Intervals brings out this detail, and can be found in RA-XX.

\AnswerSection
\ans We first form a sequence $x_n$ such that $\abs{f(x_n)} > n$, claim that it has a convergent subsequence $x_{n_k} \rightarrow x_0$ and then use continuity to say $f(x_{n_k}) \rightarrow f(x_0)$. Here's the problem -- the limit point $x_0$ could be one of the endpoints $a$ or $b$, and if the domain of $f$ is $(a, b)$ then it could be that $f(x_0)$ is not defined.

Intuitively, open intervals give you leeway to make your function unbounded at the endpoints because you don't have to bring them back down. Just look at the curve of $y = \tan x$ on $(0, \frac \pi 2)$.

\ans Use the fact that $n_k \geq k$ and apply definition of convergence to $x_n$.

\ans Suppose not and a sequence $\{a_n\}$ converges to two limits $L$ and $M$ with $L < M$. Take $\epsilon = \frac{M - L}{2}$. Then an infinite tail of the sequence is smaller than $\frac{L + M}{2}$ while another is larger. Impossible.

\textbf{Aliter.} $$0 \leq \abs{L - M} = \abs{L - a_n - (M - a_n)} \leq \abs{L - a_n} + \abs{M - a_n} < \epsilon$$
Therefore $\abs{L - M} = 0$.
\end{document}