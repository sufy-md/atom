\documentclass[a4paper, 14pt]{extarticle}

\usepackage{amsmath}
\usepackage{amssymb}
\usepackage{amsthm}
\usepackage{commath}
\usepackage[margin=0.5in]{geometry}
% \usepackage{lexend}
\usepackage{microtype}
\usepackage{parskip}
\usepackage{tikz}
\usepackage{tikz-cd}
\usepackage{tkz-euclide}
\usepackage{xparse}

\usetikzlibrary{calc,angles,quotes, positioning, shapes.geometric}

\theoremstyle{definition}
\newtheorem{dfn}{Definition}
\newtheorem{clm}{Claim}
\newtheorem{asn}{Assertion}
\newtheorem{thm}{Theorem}
\newtheorem{prb}{Problem}
\newtheorem{ans}{Answer}
\newtheorem{lm}{Lemma}
\newtheorem{rmk}{Remark}
\newtheorem{crl}{Corollary}
\newtheorem{ex}{Exercise}
\newtheorem{xmp}{Example}

\newcommand{\titleheader}[1]{\begin{centering}
\begin{LARGE}
\textbf{#1}
\end{LARGE}\\
\hrulefill

\vspace{-1.25\baselineskip}
\hrulefill
\end{centering}\\\\}

\def\changemargin#1#2{\list{}{\rightmargin#2\leftmargin#1}\item[]}
\let\endchangemargin=\endlist

\NewDocumentEnvironment{smrg}{}{\begin{changemargin}{0.5cm}{0.5cm}}{\end{changemargin}}

\NewDocumentEnvironment{SWP}{m m}
{%
  \vspace{-.9cm}%
  \begin{changemargin}{0.5cm}{0.5cm}%
  \noindent#1~#2
  \par
  \textbf{Proof.} 
}
{%
  \qed
  \end{changemargin}
}

\NewDocumentEnvironment{SNP}{m}
{%
  \vspace{-.9cm}%
  \begin{changemargin}{0.5cm}{0.5cm}%
  \noindent#1
}
{%
  \end{changemargin}
}

\newcommand{\bb}[1]{\mathbb{#1}}
\newcommand{\st}{\space \mid \space}
\newcommand{\union}[1]{\displaystyle\mathop{\cup}\limits_{#1}}
\newcommand{\intersect}[1]{\displaystyle\mathop{\cap}\limits_{#1}}
\newcommand{\paran}[1]{\left ( {#1} \right )}
\newcommand{\contra}{$\rightarrow\!\leftarrow$}
\newcommand{\kvec}[2]{({#1}_1 \dots {#1}_{#2})}
\newcommand{\pf}{\textbf{Proof.} }

\newcommand{\AnswerSection}{
    \newpage
    \section*{Answers to Exercises}
    \textit{The following are brief solutions or hints. You are encouraged to review the exercises before checking the answers.}
}
\begin{document}
\titleheader{Sequential Continuity of Functions}
\textbf{Prereqs} RA-07

You're familiar with continuous functions as those ``which can be drawn without lifting your pen". Of course, this is not a rigorous classification; but intuitively it makes sense. Whenever we impose a definition, we want that definition to capture our intuition.

Luckily for us, there are two ways we can do this. We'll look at the first one here.

Suppose you're drawing the curve of some continuous function $f$ with your pen, and you want this curve to be perfectly accurate. You've crossed $x = 1$, you've crossed $x = 2$, and you're now approaching $x = 3$. As you do, do you ensure that your pen also moves toward $x = 3$? Of course, you want to be as accurate as possible!

In other words, as you get closer to some $x_0$ on the $x-$axis, you also ensure you're getting closer to $f(x_0)$ on the $y-$axis. Where have we seen this idea of getting closer to a point before?

\begin{SNP}{\dfn}{We say a function $f: \Omega \rightarrow \bb R$ with $\Omega \subseteq \bb R$ is \emph{continuous at} $x$ if for every sequence $x_n$ that converges to $x$, the sequence $f(x_n)$ converges to $f(x)$}
\end{SNP}

In other words, $f$ is continuous at $x$ if it preserves convergence at $x$.

Observe that continuity is a pointwise property. Let me demonstrate with an example. Consider the function $f: \bb R \rightarrow \bb R$ given by
$$
f(x) = \begin{cases}
	0 & x \in \bb Q\\
	x & x \notin \bb Q
\end{cases}
$$
and consider a sequence $x_n \rightarrow 0$. Any given $x_n$ can either be rational or irrational, so $f(x_n)$ is either $x_n$ or $0$. But since both $x_n \rightarrow 0$ and the constant sequence $0 \rightarrow 0$, we conclude that $x_n \rightarrow 0 \implies f(x_n) \rightarrow 0$ and therefore $f$ is continuous at $0$.

How about some other value? If $x$ is an irrational number, we can approximate it with a rational sequence, in which case we'll get $f(x) = x$ but $f(x_n) \rightarrow 0 \neq x$, so $f$ is not continuous at any irrational.

In fact, two can play at this game. If $x$ is a nonzero rational, we can instead approximate it with irrational numbers. In that case, $f(x) = 0$ but $f(x_n) \rightarrow x \neq 0$. Again, $f$ is not continuous at any nonzero rational.

So $f$ is only continuous at one point! Absurd, isn't it? We started out with a reasonable quantification of our pen and paper idea of continuity, and it says that a function we can't even draw is continuous at a lone point?

As it turns out, this definition of continuity (called \emph{sequential continuity}) is perfectly fine, and is \emph{equivalent} to another definition that appears ``more correct'' on the surface. Our intuition wasn't wrong, the definition and its consequences just tell us what our intuition failed to capture, and that's okay.
\end{document}