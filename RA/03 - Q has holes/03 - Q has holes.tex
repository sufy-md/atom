\documentclass[a4paper, 14pt]{extarticle}

\usepackage{amsmath}
\usepackage{amssymb}
\usepackage{amsthm}
\usepackage{commath}
\usepackage[margin=0.5in]{geometry}
% \usepackage{lexend}
\usepackage{microtype}
\usepackage{parskip}
\usepackage{tikz}
\usepackage{tikz-cd}
\usepackage{tkz-euclide}
\usepackage{xparse}

\usetikzlibrary{calc,angles,quotes, positioning, shapes.geometric}

\theoremstyle{definition}
\newtheorem{dfn}{Definition}
\newtheorem{clm}{Claim}
\newtheorem{asn}{Assertion}
\newtheorem{thm}{Theorem}
\newtheorem{prb}{Problem}
\newtheorem{ans}{Answer}
\newtheorem{lm}{Lemma}
\newtheorem{rmk}{Remark}
\newtheorem{crl}{Corollary}
\newtheorem{ex}{Exercise}
\newtheorem{xmp}{Example}

\newcommand{\titleheader}[1]{\begin{centering}
\begin{LARGE}
\textbf{#1}
\end{LARGE}\\
\hrulefill

\vspace{-1.25\baselineskip}
\hrulefill
\end{centering}\\\\}

\def\changemargin#1#2{\list{}{\rightmargin#2\leftmargin#1}\item[]}
\let\endchangemargin=\endlist

\NewDocumentEnvironment{smrg}{}{\begin{changemargin}{0.5cm}{0.5cm}}{\end{changemargin}}

\NewDocumentEnvironment{SWP}{m m}
{%
  \vspace{-.9cm}%
  \begin{changemargin}{0.5cm}{0.5cm}%
  \noindent#1~#2
  \par
  \textbf{Proof.} 
}
{%
  \qed
  \end{changemargin}
}

\NewDocumentEnvironment{SNP}{m}
{%
  \vspace{-.9cm}%
  \begin{changemargin}{0.5cm}{0.5cm}%
  \noindent#1
}
{%
  \end{changemargin}
}

\newcommand{\bb}[1]{\mathbb{#1}}
\newcommand{\st}{\space \mid \space}
\newcommand{\union}[1]{\displaystyle\mathop{\cup}\limits_{#1}}
\newcommand{\intersect}[1]{\displaystyle\mathop{\cap}\limits_{#1}}
\newcommand{\paran}[1]{\left ( {#1} \right )}
\newcommand{\contra}{$\rightarrow\!\leftarrow$}
\newcommand{\kvec}[2]{({#1}_1 \dots {#1}_{#2})}
\newcommand{\pf}{\textbf{Proof.} }

\newcommand{\AnswerSection}{
    \newpage
    \section*{Answers to Exercises}
    \textit{The following are brief solutions or hints. You are encouraged to review the exercises before checking the answers.}
}
\begin{document}
\titleheader{$\bb Q$ has holes}
\textbf{Prereqs} RA-02

It is well known that the square root of $2$ is not a rational number. In particular, one can say $\bb Q$ has a \emph{hole} at $\sqrt 2$. We will try to capture this idea in mathematical terms. i.e., we will try to rigorously define what exactly constitutes a \emph{hole}.

Consider $S = \{x \in \bb Q \st x \geq 0 \text{ and } x^2 < 2\}$. This is clearly bounded above by $2$, for $x^2 < 2$ implies $x^2 < 4$, in turn implies $x < 2$. We don't worry about absolute value when taking square roots in the last step because $x \geq 0$.

Recall in the examples in RA-02, we were able to find some number $\alpha$ such that nothing smaller than $\alpha$ was an upper bound of the set. Can we do the same here?

As it turns out, the answer is no, we can't. In fact, we can \emph{prove} that $S$ has no least upper bound in $\bb Q$

\begin{SWP}{\lm}{$S$ has no least upper bound in $\bb Q$}
Let $p \in \bb Q$ be so that $p \geq 0$. We will observe a few things about $p$.

First, $p^2 \neq 2$. In the case that $p^2 < 2$, $p \in S$. In the case that $p^2 > 2$, $p$ is an upper bound of $S$.

We wish to show $p$ is not a least upper bound of $S$. We do so by exhibiting another upper bound of $S$ which is smaller than $p$. Let
$$
q = p - \dfrac{p^2 - 2}{p + 2} = \dfrac{2p + 2}{p + 2}
$$

Since $p^2 - 2 > 0$, we subtract a positive quantity from $p$ and thus $q < p$. $q$ is clearly $> 0$ because $p > 0$. Now observe
\begin{align*}
q^2 - 2 &= \dfrac{(2p+2)^2}{(p+2)^2} - 2\\
		&= \dfrac{4p^2 + 8p + 4}{p^2 + 4p + 4} - 2\\
		&= \dfrac{4p^2 + 8p + 4 - 2p^2 - 8p - 8}{(p + 2)^2}\\
		&= \dfrac{2(p^2 - 2)}{(p + 2)^2}
\end{align*}
Again since $p^2 - 2 > 0$, we get that $q^2 - 2 > 0$, so $q$ is an upper bound of $S$!
\end{SWP}

Here, have some intuition. Consider the interval $[0, \sqrt 2]$ as a subset of $\bb R$. I know the least upper bound $\alpha$ has the property that any $x$ smaller than it satisfies $x^2 < 2$ (compare to above) and any $x$ greater than it satisfies $x^2 > 2$. The least upper bound is thus \emph{forced} to satisfy $x^2 = 2$. Since no such number exists in $\bb Q$, there's a \emph{hole} there.

\end{document}