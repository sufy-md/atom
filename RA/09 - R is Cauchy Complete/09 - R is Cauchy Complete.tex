\documentclass[a4paper, 14pt]{extarticle}

\usepackage{amsmath}
\usepackage{amssymb}
\usepackage{amsthm}
\usepackage{commath}
\usepackage[margin=0.5in]{geometry}
% \usepackage{lexend}
\usepackage{microtype}
\usepackage{parskip}
\usepackage{tikz}
\usepackage{tikz-cd}
\usepackage{tkz-euclide}
\usepackage{xparse}

\usetikzlibrary{calc,angles,quotes, positioning, shapes.geometric}

\theoremstyle{definition}
\newtheorem{dfn}{Definition}
\newtheorem{clm}{Claim}
\newtheorem{asn}{Assertion}
\newtheorem{thm}{Theorem}
\newtheorem{prb}{Problem}
\newtheorem{ans}{Answer}
\newtheorem{lm}{Lemma}
\newtheorem{rmk}{Remark}
\newtheorem{crl}{Corollary}
\newtheorem{ex}{Exercise}
\newtheorem{xmp}{Example}

\newcommand{\titleheader}[1]{\begin{centering}
\begin{LARGE}
\textbf{#1}
\end{LARGE}\\
\hrulefill

\vspace{-1.25\baselineskip}
\hrulefill
\end{centering}\\\\}

\def\changemargin#1#2{\list{}{\rightmargin#2\leftmargin#1}\item[]}
\let\endchangemargin=\endlist

\NewDocumentEnvironment{smrg}{}{\begin{changemargin}{0.5cm}{0.5cm}}{\end{changemargin}}

\NewDocumentEnvironment{SWP}{m m}
{%
  \vspace{-.9cm}%
  \begin{changemargin}{0.5cm}{0.5cm}%
  \noindent#1~#2
  \par
  \textbf{Proof.} 
}
{%
  \qed
  \end{changemargin}
}

\NewDocumentEnvironment{SNP}{m}
{%
  \vspace{-.9cm}%
  \begin{changemargin}{0.5cm}{0.5cm}%
  \noindent#1
}
{%
  \end{changemargin}
}

\newcommand{\bb}[1]{\mathbb{#1}}
\newcommand{\st}{\space \mid \space}
\newcommand{\union}[1]{\displaystyle\mathop{\cup}\limits_{#1}}
\newcommand{\intersect}[1]{\displaystyle\mathop{\cap}\limits_{#1}}
\newcommand{\paran}[1]{\left ( {#1} \right )}
\newcommand{\contra}{$\rightarrow\!\leftarrow$}
\newcommand{\kvec}[2]{({#1}_1 \dots {#1}_{#2})}
\newcommand{\pf}{\textbf{Proof.} }

\newcommand{\AnswerSection}{
    \newpage
    \section*{Answers to Exercises}
    \textit{The following are brief solutions or hints. You are encouraged to review the exercises before checking the answers.}
}
\begin{document}
\titleheader{R is Cauchy Complete}
\textbf{Prereqs} RA-07

We will now revisit an important theorem left without proof in RA-07. This is important enough to be seperated from other sequential properties despite being one.

\begin{SNP}{\thm}Every Cauchy sequence in $\bb R$ converges
\end{SNP}

We will need two lemmas. Once I state them, it should be immediately clear how the proof follows. If it isn't, you need to revisit RA-08.

\begin{SWP}{\lm}{Every Cauchy sequence is bounded}We'll proceed similar to the proof that every convergent sequence is bounded. Let $\{a_n\}$ be Cauchy and fix $\epsilon_0 = 1$. Then some $N \in \bb N$ is such that whenever $m, n \geq N$ then $\abs{a_m - a_n} < \epsilon_0$, by definition of Cauchy sequence.

If we fix $n = N$, then for every $m \geq N$ we have $\abs{a_m - a_N} < \epsilon_0$. Thus, if $k \geq N$, then $\abs{a_k} \leq \epsilon_0 + \abs{a_N}$ by triangle inequality. i.e, $a_N, a_{N + 1}, a_{N + 2} \dots$ are all bounded. Since the remaining terms are all finitely many, they are also bounded. Therefore the entire sequence is bounded.
\end{SWP}
\begin{SWP}{\lm}{Let $\{a_n\}$ be a Cauchy sequence and let $\{a_{n_k}\}$ be a subsequence such that $a_{n_k} \rightarrow L$. Then, $a_n \rightarrow L$}We want to show $\{a_n\}$ converges to $L$. Fix some $\epsilon > 0$. We apply the definition of Cauchy sequence to $\{a_n\}$ and convergent sequence to $\{a_{n_k}\}$ with $\dfrac{\epsilon}{2}$.

Since $\{a_n\}$ is Cauchy, there is some $N_1$ such that whenever $m, n \geq N_1$, we get $\abs{a_n - a_m} < \dfrac{\epsilon}{2}$

Since $a_{n_k}$ converges, there is some $N_2$ such that whenever $n_k \geq N_2$, we get $\abs{a_{n_k} - L} < \dfrac{\epsilon}{2}$

Take $N_3 = \max \{N_1, N_2\}$. Fix $k > N_3$, then certainly $n_k > N_3$, and for any $n > k$ we get
\begin{align*}
\abs{a_n - L} &= \abs{a_n - a_{n_k} - (L - {a_{n_k}})}\\
			  &\leq \abs{a_n - a_{n_k}} + \abs{L - a_{n_k}} \text{ (triangle inequality)}\\
			  &\leq \dfrac{\epsilon}{2} + \dfrac{\epsilon}{2} \text{ (since } n, n_k \geq N_1 \text{ and } a_n \text{ is Cauchy; $n_k \geq N_1$ and $a_{n_k}$ converges)}\\
			  &= \epsilon
\end{align*}
Therefore, if a Cauchy sequence has a convergent subsequence, the original sequence also converges to the same limit.
\end{SWP}

Finally, let $\{a_n\}$ be Cauchy, thus it is bounded. Thus it has a convergent subsequence. Thus it converges.\qed
\end{document}