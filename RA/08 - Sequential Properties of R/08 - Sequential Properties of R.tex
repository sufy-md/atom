\documentclass[a4paper, 14pt]{extarticle}

\usepackage{amsmath}
\usepackage{amssymb}
\usepackage{amsthm}
\usepackage{commath}
\usepackage[margin=0.5in]{geometry}
% \usepackage{lexend}
\usepackage{microtype}
\usepackage{parskip}
\usepackage{tikz}
\usepackage{tikz-cd}
\usepackage{tkz-euclide}
\usepackage{xparse}

\usetikzlibrary{calc,angles,quotes, positioning, shapes.geometric}

\theoremstyle{definition}
\newtheorem{dfn}{Definition}
\newtheorem{clm}{Claim}
\newtheorem{asn}{Assertion}
\newtheorem{thm}{Theorem}
\newtheorem{prb}{Problem}
\newtheorem{ans}{Answer}
\newtheorem{lm}{Lemma}
\newtheorem{rmk}{Remark}
\newtheorem{crl}{Corollary}
\newtheorem{ex}{Exercise}
\newtheorem{xmp}{Example}

\newcommand{\titleheader}[1]{\begin{centering}
\begin{LARGE}
\textbf{#1}
\end{LARGE}\\
\hrulefill

\vspace{-1.25\baselineskip}
\hrulefill
\end{centering}\\\\}

\def\changemargin#1#2{\list{}{\rightmargin#2\leftmargin#1}\item[]}
\let\endchangemargin=\endlist

\NewDocumentEnvironment{smrg}{}{\begin{changemargin}{0.5cm}{0.5cm}}{\end{changemargin}}

\NewDocumentEnvironment{SWP}{m m}
{%
  \vspace{-.9cm}%
  \begin{changemargin}{0.5cm}{0.5cm}%
  \noindent#1~#2
  \par
  \textbf{Proof.} 
}
{%
  \qed
  \end{changemargin}
}

\NewDocumentEnvironment{SNP}{m}
{%
  \vspace{-.9cm}%
  \begin{changemargin}{0.5cm}{0.5cm}%
  \noindent#1
}
{%
  \end{changemargin}
}

\newcommand{\bb}[1]{\mathbb{#1}}
\newcommand{\st}{\space \mid \space}
\newcommand{\union}[1]{\displaystyle\mathop{\cup}\limits_{#1}}
\newcommand{\intersect}[1]{\displaystyle\mathop{\cap}\limits_{#1}}
\newcommand{\paran}[1]{\left ( {#1} \right )}
\newcommand{\contra}{$\rightarrow\!\leftarrow$}
\newcommand{\kvec}[2]{({#1}_1 \dots {#1}_{#2})}
\newcommand{\pf}{\textbf{Proof.} }

\newcommand{\AnswerSection}{
    \newpage
    \section*{Answers to Exercises}
    \textit{The following are brief solutions or hints. You are encouraged to review the exercises before checking the answers.}
}
\begin{document}
\titleheader{Sequential Properties of R}
\textbf{Prereqs} RA-06, RA-07

We will now move forward to some properties of real-valued sequences. It is worth noting these properties are closely related to the completeness axiom, as you can find in RA-XX

We will first look at a useful theorem.
\begin{SWP}{\thm}{(Nested Intervals) Let $\{I_n\}_n$ be a sequence of intervals $I_n = [a_n, b_n]$ such that $I_{n + 1} \subset I_{n}$ and $b_n - a_n \rightarrow 0$. Then, there is some $x$ such that $\cap I_n = \{x\}$}Observe that in other words, if we have a decreasing sequence of intervals whose lengths coverge to $0$, then the intersection of these intervals has one and only one point. Yet again, in the rationals, where this point is supposed to be might have a hole.

First we show that the intersection is nonempty. Consider the sequences of the left and right endpoints of the intervals, viz. $\{a_n\}_n$ and $\{b_n\}_n$. Notice that $\{a_n\}$ is monotone increasing, and $\{b_n\}$ is monotone decreasing. (that is, $a_1 \leq a_2 \leq a_3 \leq \dots$ and $\dots \leq b_3 \leq b_2 \leq b_1$)  

We also know that $a_k \leq b_k$ for every fixed $k$. Thus for any $m$ we have
$$
a_1 \leq a_m \leq b_m \leq b_1
$$
That is, $\{a_n\}$ is bounded above by $b_1$ and $\{b_n\}$ is bounded below by $a_1$.

By monotone convergence theorem, $a_n \rightarrow A$ and $b_n \rightarrow B$. We know $b_n - a_n \rightarrow 0$ so $B - A = 0$, i.e. $B = A =: L$ (say). We claim that this $L$ lies in the intersection and is the only point that does.

To show that $L$ lies in the intersection is straightforward. We need to show $a_k \leq L \leq b_k$ for every $k$.

From the proof of monotone convergence theorem (see RA-07) we know $L$ is the least upper bound of $\{a_n\}$ and thus $a_k \leq L$ for all $k$. Similarly $L$ is also the greatest lower bound of $\{b_n\}$ and thus $L \leq b_k$ for every $k$.

Now suppose some $L + \epsilon$ lies in the intersection with $\epsilon > 0$ (we can proceed in a similar manner for $\epsilon < 0$). If it lies in the intersection, then $L + \epsilon \leq b_k$ for every $k$, but then that would make $L + \epsilon$ a lower bound of $\{b_n\}$, when $L$ was the greatest lower bound. We have a contradiction!\contra

Therefore$$\intersect{n \in \bb N}I_n = \{L\}$$as required.
\end{SWP}
\newpage
We will now have a look at another interesting property of $\bb R$.
\begin{SNP}{\thm}{(Bolzano-Weierstrass) Every bounded sequence has a convergent subsequence}
\end{SNP}
The theorem for now will be treated as a direct consequence of the following theorem, however it admits another proof using Nested Intervals which I really encourage you to read (again, it will be in RA-XX).
\begin{SWP}{\thm}{(Monotone Subsequence) Every sequence has a monotone subsequence}Consider a sequence $\{a_n\}$. We call a term $\{a_k\}$ a \emph{peak} if $a_{k + 1}, a_{k + 2} \dots$ are all smaller than $a_k$.

Suppose first that our sequence has infinitely many peaks, $a_{n_1}, a_{n_2} \dots$ and so on. Since $a_{n_1}$ is a peak, we have $a_{n_1} \geq a_{n_2}$. Since $a_{n_2}$ is a peak, we have $a_{n_2} \geq a_{n_3}$. Inductively, we obtain
$$
a_{n_1} \geq a_{n_2} \geq a_{n_3} \geq \dots
$$
and hence a monotone decreasing subsequence.

Now suppose otherwise that there are finitely many peaks with the last being at $a_N$ (if there are no peaks, consider the last peak to be at $a_0$)

Consider the term $a_{n_1} := a_{N+1}$. This is not a peak since it is after the last peak. Thus there is some term $a_{n_2}$ after it such that $a_{n_2} \geq a_{n_1}$.

But again, $a_{n_2}$ is not a peak, so there is some $a_{n_3}$ after it such that $a_{n_3} \geq a_{n_2}$. Inductively, we get
$$
a_{n_1} \leq a_{n_2} \leq a_{n_3} \leq \dots
$$
and hence a monotone increasing subsequence.
\end{SWP}
\begin{SWP}{\crl}{Bolzano-Weierstrass Theorem}Let $\{a_n\}$ be bounded. By monotone subsequence theorem, let $\{a_{n_k}\}$ be a monotone subsequence. This is also bounded. By monotone convergence theorem, $a_{n_k}$ converges.
\end{SWP}
\end{document}