\documentclass[a4paper, 14pt]{extarticle}

\usepackage{amsmath}
\usepackage{amssymb}
\usepackage{amsthm}
\usepackage{commath}
\usepackage[margin=0.5in]{geometry}
% \usepackage{lexend}
\usepackage{microtype}
\usepackage{parskip}
\usepackage{tikz}
\usepackage{tikz-cd}
\usepackage{tkz-euclide}
\usepackage{xparse}

\usetikzlibrary{calc,angles,quotes, positioning, shapes.geometric}

\theoremstyle{definition}
\newtheorem{dfn}{Definition}
\newtheorem{clm}{Claim}
\newtheorem{asn}{Assertion}
\newtheorem{thm}{Theorem}
\newtheorem{prb}{Problem}
\newtheorem{ans}{Answer}
\newtheorem{lm}{Lemma}
\newtheorem{rmk}{Remark}
\newtheorem{crl}{Corollary}
\newtheorem{ex}{Exercise}
\newtheorem{xmp}{Example}

\newcommand{\titleheader}[1]{\begin{centering}
\begin{LARGE}
\textbf{#1}
\end{LARGE}\\
\hrulefill

\vspace{-1.25\baselineskip}
\hrulefill
\end{centering}\\\\}

\def\changemargin#1#2{\list{}{\rightmargin#2\leftmargin#1}\item[]}
\let\endchangemargin=\endlist

\NewDocumentEnvironment{smrg}{}{\begin{changemargin}{0.5cm}{0.5cm}}{\end{changemargin}}

\NewDocumentEnvironment{SWP}{m m}
{%
  \vspace{-.9cm}%
  \begin{changemargin}{0.5cm}{0.5cm}%
  \noindent#1~#2
  \par
  \textbf{Proof.} 
}
{%
  \qed
  \end{changemargin}
}

\NewDocumentEnvironment{SNP}{m}
{%
  \vspace{-.9cm}%
  \begin{changemargin}{0.5cm}{0.5cm}%
  \noindent#1
}
{%
  \end{changemargin}
}

\newcommand{\bb}[1]{\mathbb{#1}}
\newcommand{\st}{\space \mid \space}
\newcommand{\union}[1]{\displaystyle\mathop{\cup}\limits_{#1}}
\newcommand{\intersect}[1]{\displaystyle\mathop{\cap}\limits_{#1}}
\newcommand{\paran}[1]{\left ( {#1} \right )}
\newcommand{\contra}{$\rightarrow\!\leftarrow$}
\newcommand{\kvec}[2]{({#1}_1 \dots {#1}_{#2})}
\newcommand{\pf}{\textbf{Proof.} }

\newcommand{\AnswerSection}{
    \newpage
    \section*{Answers to Exercises}
    \textit{The following are brief solutions or hints. You are encouraged to review the exercises before checking the answers.}
}
\begin{document}
\titleheader{$\bb R$ is complete}
\textbf{Prereqs} RA-03, RA-02

We are now ready to state a very fundamental property of $\bb R$.
\begin{SNP}{\thm}{$\bb R$ is complete.}
\end{SNP}
The proof is irrelevant and depends on the construction of $\bb R$, for which I'll leave an outline. It can also be found in \cite{Rudin}. Let's see what completeness allows us to do.
\begin{SNP}{\thm}{Let $x \in \bb{R}$ be such that $x > 0$ and let $n \in \bb{N}$. Then, there is some $y \in \bb{R}$ such that $y^n = x$ and $y > 0$}
\end{SNP}
The theorem states that every positive number admits an $n-$th root, whenever $n$ is natural. This already is something that we cannot say about $\bb{Q}$.

Let's try to intuitively understand this. We know that $x^n$ is increasing for $n \in \bb{N}$. Suppose I know $y_1^n < x$, so my required $y$ is certainly larger than $y_1$. Suppose I also know $y_2^n > x$, so my required $y$ is certainly smaller than $y_2$. Now I can take the average of $y_1$ and $y_2$ and repeat this all over again.

In formal terms, let $S := \{y \in \bb R_{\geq 0} \st y^n \leq x\}$. Certainly, $x + 1$ is an upper bound of $S$. Also $0 \in S$. $S$ thus admits a least upper bound $\alpha$.

Any $\beta$ smaller than $\alpha$ is already in $S$ and thus $\beta^n < x$. Surprisingly, this goes the other way, if some $\beta$ satisfies $\beta ^ n < x$, I can be sure that $\beta < \alpha$ ($x^n$ is an increasing function!). So $\alpha^n$ cannot be smaller than $x$.

What if $\beta > \alpha$? Any such $\beta$ is an upper bound of $S$. Take $\gamma = \frac{\alpha + \beta}{2}$. We know $\beta ^n$ cannot be smaller than $x$, otherwise $\beta \in S$ and it will contradict that $\alpha$ is an upper bound. If $\beta^n = x$, then $\gamma^n$ must be smaller than $x$ and $\gamma$ must lie in $S$, again contradicting that $\alpha$ is an upper bound of $S$. We are forced to accept that $\beta > \alpha \iff \beta^n > x$.

What about $\alpha^n$? We already knew $\alpha^n$ cannot be smaller than $x$. From the preceding paragraph, $\alpha^n > x$ would mean $\alpha > \alpha$ which cannot happen.

It must be the case that $\alpha^n = x$. 

This proof is handwavy, the details can be found in \cite{Rudin}. The overall idea should be clear. We recognise that $x^n$ is an increasing function; so on the number line, our required $n-$th root must be sandwiched between those $y$ for which $y^n < x$, and those $y$ for which $y^n > x$.

On the rational number line, there might be a hole where this sandwiched number is supposed to be. The completeness property ensures no such hole exists.
\bibliography{../../refs}
\bibliographystyle{plain}
\end{document}