\documentclass[a4paper, 14pt]{extarticle}

\usepackage{amsmath}
\usepackage{amssymb}
\usepackage{amsthm}
\usepackage{commath}
\usepackage[margin=0.5in]{geometry}
% \usepackage{lexend}
\usepackage{microtype}
\usepackage{parskip}
\usepackage{tikz}
\usepackage{tikz-cd}
\usepackage{tkz-euclide}
\usepackage{xparse}

\usetikzlibrary{calc,angles,quotes, positioning, shapes.geometric}

\theoremstyle{definition}
\newtheorem{dfn}{Definition}
\newtheorem{clm}{Claim}
\newtheorem{asn}{Assertion}
\newtheorem{thm}{Theorem}
\newtheorem{prb}{Problem}
\newtheorem{ans}{Answer}
\newtheorem{lm}{Lemma}
\newtheorem{rmk}{Remark}
\newtheorem{crl}{Corollary}
\newtheorem{ex}{Exercise}
\newtheorem{xmp}{Example}

\newcommand{\titleheader}[1]{\begin{centering}
\begin{LARGE}
\textbf{#1}
\end{LARGE}\\
\hrulefill

\vspace{-1.25\baselineskip}
\hrulefill
\end{centering}\\\\}

\def\changemargin#1#2{\list{}{\rightmargin#2\leftmargin#1}\item[]}
\let\endchangemargin=\endlist

\NewDocumentEnvironment{smrg}{}{\begin{changemargin}{0.5cm}{0.5cm}}{\end{changemargin}}

\NewDocumentEnvironment{SWP}{m m}
{%
  \vspace{-.9cm}%
  \begin{changemargin}{0.5cm}{0.5cm}%
  \noindent#1~#2
  \par
  \textbf{Proof.} 
}
{%
  \qed
  \end{changemargin}
}

\NewDocumentEnvironment{SNP}{m}
{%
  \vspace{-.9cm}%
  \begin{changemargin}{0.5cm}{0.5cm}%
  \noindent#1
}
{%
  \end{changemargin}
}

\newcommand{\bb}[1]{\mathbb{#1}}
\newcommand{\st}{\space \mid \space}
\newcommand{\union}[1]{\displaystyle\mathop{\cup}\limits_{#1}}
\newcommand{\intersect}[1]{\displaystyle\mathop{\cap}\limits_{#1}}
\newcommand{\paran}[1]{\left ( {#1} \right )}
\newcommand{\contra}{$\rightarrow\!\leftarrow$}
\newcommand{\kvec}[2]{({#1}_1 \dots {#1}_{#2})}
\newcommand{\pf}{\textbf{Proof.} }

\newcommand{\AnswerSection}{
    \newpage
    \section*{Answers to Exercises}
    \textit{The following are brief solutions or hints. You are encouraged to review the exercises before checking the answers.}
}
\begin{document}
\titleheader{Taylor's Theorem II}
\textbf{Prereqs} RA-16

The information about how a function $f$ twists and turns near a point $x_0$ is stored in the derivatives $f'(x_0), f''(x_0) \ldots f^{(k)}(x_0) \ldots$ if they exist. Recall our second-order Extended Lagrange approximation
$$
f(x) \approx f(a) + (x - a)\cdot f'(a) + \dfrac{(x - a)^2}{2}\cdot f''(a)
$$
And observe that the RHS is a $2-$degree polynomial.
$$
P_2(x) = f(a) + (x - a)\cdot f'(a) + \dfrac{(x - a)^2}{2}\cdot f''(a)
$$
Here's what makes $P_2$ interesting. The polynomial and its derivatives upto order $2$ agree with $f$. Note that
\begin{align*}
P_2(x) &= f(a) + (x - a)\cdot f'(a) + \dfrac{(x - a)^2}{2}\cdot f''(a)\\
P_2'(x) &= f'(a) + (x - a)\cdot f''(a)\\
P_2''(x) &= f''(a)
\end{align*}
Hence at the point $x = a$,
\begin{align*}
P_2(a) &= f(a)\\
P_2'(a) &= f'(a)\\
P_2''(a) &= f''(a)
\end{align*}
i.e., the $0-$th, $1-$st and $2-$nd order curvatures of $P_2$ at $a$ agree precisely with that of $f$, so long as $f$ admits those derivatives.

We're ready to ask the general question. Suppose $f$ is $k-$times differentiable. Does $f$ admit a $k-$degree polynomial $P_k$ such that
$$
f^{(m)}(a) = P_k^{(m)}(a)
$$
for every $0 \leq m \leq k$? The answer is yes, and the construction is outlined below.

Suppose the following form for the required polynomial.
$$
P_k(x) = a_0 + a_1\cdot (x - a) + a_2\cdot (x - a)^2 + \ldots + a_k\cdot (x - a)^k
$$
$P_k(a) = f(a)$ immediately forces
$$
a_0 = f(a)
$$
Now consider some $m$ with $1 \leq m \leq k$ and consider $P_k^{(m)}(x)$. To know what $P_k^{(m)}$ looks like, it suffices to find out what happens to the term $a_n(x - a)^n$ when it is differentiated $m$ times.

It is a rather simple verification that\begin{align*}(a_n(x - a)^n)^{(m)} &= a_n\cdot n \cdot (n - 1) \ldots (n + 1 - m)(x - a)^{n - m}\end{align*}
Which simplifies as follows$$
\begin{array}{l}
n < m \text{ : } 0\\
n = m \text{ : } a_m\cdot m!\\
n > m \text{ : } \dfrac{a_n\cdot n!}{(n - m)!}(x - a)^{n - m}
\end{array}
$$
Therefore $P_k^{(m)}$ is given by
$$
P_k^{(m)}(x) = a_m\cdot m! + \sum\limits_{n = m + 1}^{k}\dfrac{a_n \cdot n!}{(n - m)!}(x - a)^{n - m}
$$
which gives
$$
P_k^{(m)}(a) = a_m\cdot m!
$$
Here we enforce $f^{(m)}(a) = P_k^{(m)}(a)$. Hence,
$$
a_m = \dfrac{f^{(m)}(a)}{m!}
$$
Thus finally we obtain $P_k^{(m)}$
\begin{align*}
P_k^{(m)}(x) &= f(a) + (x - a)\cdot f'(a) + \dfrac{(x - a)^2}{2!}\cdot f''(a) + \ldots \\
			 &= \sum\limits_{n = 0}^{k}\dfrac{f^{(n)}(a)}{n!}\cdot (x - a)^n
\end{align*}
Do you now see why there was a $2$ in the denominator alongside $h^2$ in RA-16?

Now we ask the main question. Suppose $f$ is differentiable $k$ times near $a$. Then it admits a natural approximation near $a$ in the form of $P_k(x)$. We know that degree $1$ and degree $2$ approximations of this kind are in fact precise, if we just adjust the last term. Is this true for a general $k-$degree approximation?

The answer is in the affirmative, and the proof mimics that of Extended Lagrange in RA-16. The precise formulation is provided below.
\begin{SWP}{\thm}{(Taylor-Lagrange) Let $f:[a, b] \rightarrow \bb R$ be $k-$times differentiable on $(a, b)$, and let $f ,f', f'' \ldots f^{(k - 1)}$ be continuous on $[a, b]$. Then there is $c \in (a, b)$ such that
$$
f(b) = f(a) + (b - a)\cdot f'(a) + \ldots + \dfrac{(b - a)^{k - 1}}{(k - 1)!}\cdot f^{(k - 1)}(a) + \dfrac{(b - a)^k}{k!}\cdot f^{(k)}(c)
$$}We will mimic the proof provided in RA-16.

Define
\begin{align*}	
F(x) &= f(b) - f(x) - (b - x)\cdot f'(x) - \ldots - \dfrac{(b - x)^{k - 1}}{(k - 1)!}\cdot f^{(k - 1)}(x)\\
     &= f(b) - f(x) - \sum\limits_{n = 1}^{k - 1}\dfrac{(b - x)^n}{n!}\cdot f^{(n)}(x)
\end{align*}
Now to take the derivative
\begin{align*}
F'(x) = - f'(x) - \sum\limits_{n = 1}^{k - 1}\paran{\dfrac{(b - x)^n}{n!}\cdot f^{(n)}(x)}'
\end{align*}
we need to use the product rule for the second term. The $n-$th term breaks into
$$
\dfrac{(b - x)^n}{n!}\cdot f^{(n + 1)}(x) - \dfrac{(b - x)^{n - 1}}{(n - 1)!}\cdot f^{(n)}(x)
$$
where the $-$ sign shows up because of the chain rule on the $(b - x)$ term. The derivative then becomes
$$
F'(x) = -f'(x) - \paran{\sum\limits_{n = 1}^{k - 1}\dfrac{(b - x)^n}{n!}\cdot f^{(n + 1)}(x)} + \paran{\sum\limits_{n = 1}^{k - 1}\dfrac{(b - x)^{n - 1}}{(n - 1)!}\cdot f^{(n)}(x)}
$$
The sharp eyed will notice that a reindexing on the first sum will simplify the expression.
$$
F'(x) = -f'(x) - \paran{\sum\limits_{n = 2}^{k}\dfrac{(b - x)^{n - 1}}{(n - 1)!}\cdot f^{(n)}(x)} + \paran{\sum\limits_{n = 1}^{k - 1}\dfrac{(b - x)^{n - 1}}{(n - 1)!}\cdot f^{(n)}(x)}
$$
Only the $n = k$ term in the first sum and the $n = 1$ term in the second sum survive, and the rest get cancelled. Thus
\begin{align*}
F'(x) &= - f'(x) - \dfrac{(b - x)^{k - 1}}{(k - 1)!}\cdot f^{(k)}(x) + f'(x)\\
      &= - \dfrac{(b - x)^{k - 1}}{(k - 1)!}\cdot f^{(k)}(x)
\end{align*}
As before, $F(b) = 0$. Now if we let
$$
g(x) = F(x) - \paran{\dfrac{b - x}{b - a}}^k\cdot F(a)
$$
Then $g(a) = g(b) = 0$. Thus by Rolle's Theorem $g'(c) = 0$ for some $c \in (a, b)$. Differentiating $g$ we get
$$
g'(x) = F'(x) + k\cdot \dfrac{(b - x)^{k - 1}}{(b - a)^k}\cdot F(a)
$$
Substituting $g'(c) = 0$ and $F'(x)$ from above,
\begin{align*}
0 = -\dfrac{(b - c)^{k - 1}}{(k - 1)!}\cdot f^{(k)}(c) + k\cdot \dfrac{(b - c)^{k - 1}}{(b - a)^k}\cdot F(a)
\end{align*}
Transposing,
\begin{align*}
\dfrac{(b - c)^{k - 1}}{(k - 1)!}\cdot f^{(k)}(c) = k\cdot \dfrac{(b - c)^{k - 1}}{(b - a)^k}\cdot F(a)
\end{align*}
Since $c \in (a, b)$, $b - c \neq 0$ and we can safely remove it. We also know $k\cdot (k - 1)! = k!$. Thus,
$$
F(a) = \dfrac{(b - a)^k}{k!}\cdot f^{(k)}(c)
$$
Substituting back $F(a)$ from its definition,
$$
f(b) = f(a) + (b - a)\cdot f'(a) + \ldots + \dfrac{(b - a)^{k - 1}}{(k - 1)!}\cdot f^{(k - 1)}(a) + \dfrac{(b - a)^k}{k!}\cdot f^{(k)}(c)
$$
as stipulated.
\end{SWP}
Therefore, \emph{any} $k$ times differentiable function admits a natural $k-$degree polynomial approximation near a point of differentiability
$$
f(a + h) \approx f(a) + h\cdot f'(a) + \ldots + \dfrac{h^k}{k!}f^{(k)}(a)
$$
and the approximation is made precise by just adjusting the last term.
$$
f(a + h) = f(a) + h\cdot f'(a) + \ldots + \dfrac{h^k}{k!}f^{(k)}(c)
$$
\end{document}