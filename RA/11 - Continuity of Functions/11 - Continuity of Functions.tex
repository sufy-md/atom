\documentclass[a4paper, 14pt]{extarticle}

\usepackage{amsmath}
\usepackage{amssymb}
\usepackage{amsthm}
\usepackage{commath}
\usepackage[margin=0.5in]{geometry}
% \usepackage{lexend}
\usepackage{microtype}
\usepackage{parskip}
\usepackage{tikz}
\usepackage{tikz-cd}
\usepackage{tkz-euclide}
\usepackage{xparse}

\usetikzlibrary{calc,angles,quotes, positioning, shapes.geometric}

\theoremstyle{definition}
\newtheorem{dfn}{Definition}
\newtheorem{clm}{Claim}
\newtheorem{asn}{Assertion}
\newtheorem{thm}{Theorem}
\newtheorem{prb}{Problem}
\newtheorem{ans}{Answer}
\newtheorem{lm}{Lemma}
\newtheorem{rmk}{Remark}
\newtheorem{crl}{Corollary}
\newtheorem{ex}{Exercise}
\newtheorem{xmp}{Example}

\newcommand{\titleheader}[1]{\begin{centering}
\begin{LARGE}
\textbf{#1}
\end{LARGE}\\
\hrulefill

\vspace{-1.25\baselineskip}
\hrulefill
\end{centering}\\\\}

\def\changemargin#1#2{\list{}{\rightmargin#2\leftmargin#1}\item[]}
\let\endchangemargin=\endlist

\NewDocumentEnvironment{smrg}{}{\begin{changemargin}{0.5cm}{0.5cm}}{\end{changemargin}}

\NewDocumentEnvironment{SWP}{m m}
{%
  \vspace{-.9cm}%
  \begin{changemargin}{0.5cm}{0.5cm}%
  \noindent#1~#2
  \par
  \textbf{Proof.} 
}
{%
  \qed
  \end{changemargin}
}

\NewDocumentEnvironment{SNP}{m}
{%
  \vspace{-.9cm}%
  \begin{changemargin}{0.5cm}{0.5cm}%
  \noindent#1
}
{%
  \end{changemargin}
}

\newcommand{\bb}[1]{\mathbb{#1}}
\newcommand{\st}{\space \mid \space}
\newcommand{\union}[1]{\displaystyle\mathop{\cup}\limits_{#1}}
\newcommand{\intersect}[1]{\displaystyle\mathop{\cap}\limits_{#1}}
\newcommand{\paran}[1]{\left ( {#1} \right )}
\newcommand{\contra}{$\rightarrow\!\leftarrow$}
\newcommand{\kvec}[2]{({#1}_1 \dots {#1}_{#2})}
\newcommand{\pf}{\textbf{Proof.} }

\newcommand{\AnswerSection}{
    \newpage
    \section*{Answers to Exercises}
    \textit{The following are brief solutions or hints. You are encouraged to review the exercises before checking the answers.}
}
\begin{document}
\titleheader{Continuity of Functions}
\textbf{Prereqs} RA-10, FD-02

In RA-10 we quantified continuity of functions via sequences, the idea being that as $x_n$ get closer to some $x_0$, $f(x_n)$ should get closer to $f(x_0)$ for $f$ to be continuous.

Here, to define continuity, we impose what appears to be a stronger condition. We start with the value $f(x_0)$ and require that given any $\epsilon > 0$, there \emph{must exist} some $\delta > 0$ such that whenever $x$ is within $\delta$ of $x_0$, $f(x)$ is within $\epsilon$ of $f(x_0)$.

In other words, in the sequential definition, we started on the $x-$axis (``as $x_n$ get closer to $x_0\dots$'') and made conclusions about the $y-$axis (``...$f(x_n)$ get closer to $f(x_0)$'').

However, in this ``$\epsilon-\delta$'' definition, we \emph{start} on the $y-$axis (``given an $\epsilon-$neighborhood around $f(x_0)\dots$'') and impose a condition on the $x-$axis (``$\dots$whenever $x$ is within $\delta$ of $x_0$, $f(x)$ lands within that neighborhood'').

On first glance, the second definition looks stronger. $f(x)$ has no choice BUT to land within $\epsilon$ of $f(x_0)$ on the entire interval $(x_0 - \delta, x_0 + \delta)$. The first definition asks $f(x_n)$ to get $\epsilon-$close to $f(x_0)$ only after a certain $N$. Let's formalise and see if this is true.

\begin{SNP}{\dfn}{$f:\Omega \rightarrow \bb R$ with $\Omega \subseteq \bb R$ is \emph{continuous at} $x_0$ if for every $\epsilon > 0$, there is some $\delta > 0$ such that $x \in (x_0 - \delta, x_0 + \delta) \implies f(x) \in (f(x_0) - \epsilon, f(x_0) + \epsilon)$}
\end{SNP}

That is, you can get $f(x)$ as close to $f(x_0)$ as you want if $x$ is sufficiently close to $x_0$.

We haven't yet justified the fact that both these definitions are for $f$ to be continuous at a fixed point. We must show they are equivalent.

\begin{SWP}{\thm}{A function $f$ is continuous at $x_0$ if and only if it is sequentially continuous at $x_0$}$(\rightarrow)$ We want to show continuous implies sequentially continuous. i.e, we're given for every $\epsilon$ some $\delta$ has the property that$$\abs{x - x_0} < \delta \implies \abs{f(x) - f(x_0)} < \epsilon$$and we want to show whenever $x_n \rightarrow x_0$ then $f(x_n) \rightarrow f(x_0)$.

Let $\epsilon_0 > 0$ be given and choose $\delta_0$ by definition of continuity of $f$ at $x_0$. Suppose $x_n \rightarrow x_0$. By definition of convergence, there is some $N$ such that
$$
n \geq N \implies \abs{x_n - x_0} < \delta_0
$$
But $\delta_0$ was chosen such that
$$
\abs{x_n - x_0} < \delta_0 \implies \abs{f(x_n) - f(x_0)} < \epsilon_0
$$
Therefore whenever $n \geq N$, $\abs{f(x_n) - f(x_0)} < \epsilon_0$. By definition of convergence, $f(x_n) \rightarrow f(x_0)$\qed
\newpage
$(\leftarrow)$ Now we want to show sequentially continuous implies continuous. This is not straightforward, and instead we'll show the contrapositive -- that if $f$ is not continuous then it is not sequentially continuous. 

Suppose $f$ is not continuous. Recall that $f$ is continuous only when for every $\epsilon$, there is some $\delta$ with a certain property. For $f$ to not be continuous, there must be some $\epsilon_0$ such that every $\delta$ fails to have that property. In all, we're given that there is some $\epsilon_0$, such that for every $\delta$, there is some $x_\delta$ such that
$$
x_\delta \in (x_0 - \delta, x_0 + \delta)
$$
but
$$
f(x_\delta) \notin (f(x_0) - \epsilon_0, f(x_0) + \epsilon_0)
$$
And we want to show $f$ is not sequentially continuous -- that there is some sequence $x_n \rightarrow x_0$ but $f(x_n) \not\rightarrow f(x_0)$

Now this is straightforward. Knowing $\epsilon_0$ and that the property fails for every $\delta$, take $\delta_n = \frac{1}{n}$. Then, there is some $x_n := x_{\delta_n}$ such that
$$
x_n \in (x_0 - \dfrac 1 n, x_0 + \dfrac 1 n)
$$
but
$$
\abs{f(x_n) - f(x_0)} \geq \epsilon_0
$$
This $x_n$ is our required sequence. Clearly, $x_n \rightarrow x_0$ (since every subsequent term lies in smaller and smaller intervals around $x_0$) but $f(x_n) \not\rightarrow f(x_0)$ (since $f(x_n)$ fail to get closer to $f(x_0)$ than $\epsilon_0$).

Thus $f$ is not sequentially continuous.
\end{SWP}
Continuity and sequential continuity are the same thing! Finally, we make the following definition

\begin{SNP}{\dfn}{A function $f:\Omega \rightarrow \bb R$ is \emph{continuous} if it is \emph{continuous at} every point of $\Omega$}
\end{SNP}

Where we generalise from continuity at a point to continuity on a set.

\begin{SNP}{\ex}{Show that $f(x) = x$ is continuous.}
\end{SNP}

Recall from FD-02 the various definitions of limits. Comparing to the definition of continuity at a point, we can clearly see that the following definition of continuity is equivalent to both our above definitions.
\begin{SNP}{\dfn}{A function $f:\Omega \rightarrow \bb R$ with $x_0 \in \Omega \subseteq \bb R$ is \emph{continuous at} $x_0$ if
$$
\lim_{x\rightarrow x_0} f(x) = L
$$}
\end{SNP}
\AnswerSection
\ans Let $x_n$ be a sequence converging to an arbitrary $x_0$, i.e $x_n \rightarrow x_0$. But $f(x_n) = x_n$ and $f(x_0) = x_0$, so $f(x_n) \rightarrow f(x_0)$.\qed

\textbf{Aliter.} Let $\epsilon > 0$ be given. Take $\delta = \epsilon$ and suppose $\abs{x - x_0} < \delta$. Then,
\begin{align*}
\abs{f(x) - f(x_0)} &= \abs{x - x_0}\\
					&< \delta \\
					&= \epsilon
\end{align*}
Thus $f$ is continuous.\qed
\end{document}