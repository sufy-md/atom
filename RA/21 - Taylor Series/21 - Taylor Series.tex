\documentclass[a4paper, 14pt]{extarticle}

\usepackage{amsmath}
\usepackage{amssymb}
\usepackage{amsthm}
\usepackage{commath}
\usepackage[margin=0.5in]{geometry}
% \usepackage{lexend}
\usepackage{microtype}
\usepackage{parskip}
\usepackage{tikz}
\usepackage{tikz-cd}
\usepackage{tkz-euclide}
\usepackage{xparse}

\usetikzlibrary{calc,angles,quotes, positioning, shapes.geometric}

\theoremstyle{definition}
\newtheorem{dfn}{Definition}
\newtheorem{clm}{Claim}
\newtheorem{asn}{Assertion}
\newtheorem{thm}{Theorem}
\newtheorem{prb}{Problem}
\newtheorem{ans}{Answer}
\newtheorem{lm}{Lemma}
\newtheorem{rmk}{Remark}
\newtheorem{crl}{Corollary}
\newtheorem{ex}{Exercise}
\newtheorem{xmp}{Example}

\newcommand{\titleheader}[1]{\begin{centering}
\begin{LARGE}
\textbf{#1}
\end{LARGE}\\
\hrulefill

\vspace{-1.25\baselineskip}
\hrulefill
\end{centering}\\\\}

\def\changemargin#1#2{\list{}{\rightmargin#2\leftmargin#1}\item[]}
\let\endchangemargin=\endlist

\NewDocumentEnvironment{smrg}{}{\begin{changemargin}{0.5cm}{0.5cm}}{\end{changemargin}}

\NewDocumentEnvironment{SWP}{m m}
{%
  \vspace{-.9cm}%
  \begin{changemargin}{0.5cm}{0.5cm}%
  \noindent#1~#2
  \par
  \textbf{Proof.} 
}
{%
  \qed
  \end{changemargin}
}

\NewDocumentEnvironment{SNP}{m}
{%
  \vspace{-.9cm}%
  \begin{changemargin}{0.5cm}{0.5cm}%
  \noindent#1
}
{%
  \end{changemargin}
}

\newcommand{\bb}[1]{\mathbb{#1}}
\newcommand{\st}{\space \mid \space}
\newcommand{\union}[1]{\displaystyle\mathop{\cup}\limits_{#1}}
\newcommand{\intersect}[1]{\displaystyle\mathop{\cap}\limits_{#1}}
\newcommand{\paran}[1]{\left ( {#1} \right )}
\newcommand{\contra}{$\rightarrow\!\leftarrow$}
\newcommand{\kvec}[2]{({#1}_1 \dots {#1}_{#2})}
\newcommand{\pf}{\textbf{Proof.} }

\newcommand{\AnswerSection}{
    \newpage
    \section*{Answers to Exercises}
    \textit{The following are brief solutions or hints. You are encouraged to review the exercises before checking the answers.}
}
\begin{document}
\titleheader{Taylor Series}
\textbf{Prereqs} RA-17, RA-18, RA-20

Let $f$ be an infinitely differentiable function. More formally, suppose $f$ admits an $n-$th derivative near a point $a$ for every $n \in \bb N$. Then, $f$ admits the $n-$th Taylor Polynomial near $a$
$$
P_n(x) := f(a) + (x - a)\cdot f'(a) + \dfrac{(x - a)^2}{2}\cdot f''(a) + \dots + \dfrac{(x - a)^n}{n!}\cdot f^{(n)}(a)
$$
for every $n \in \bb N$.

In the limit as $n\to \infty$, the Taylor Polynomial $P_n$ becomes a power series around $a$ with $a_n = \frac{f^{(n)}(a)}{n!}$.
$$
P_{\infty}(x) = \sum \dfrac{f^{(n)}(a)}{n!} \cdot (x - a)^n
$$
The Taylor Polynomials $P_n$ were originally framed as polynomial approximations to differentiable functions. It is intuitive that for ``nice'' enough functions, the power series is an $\infty$-degree polynomial approximation to $f$ near $a$, and thus should be accurate.

The question we want to investigate is if the Taylor Series $P_\infty$ converges, and if it converges to $f$.

Observe that $P_n(x)$ is the sequence of partial sums of $P_\infty(x)$, and therefore by definition $P_\infty(x)$ converges iff the sequence $a_n := P_n(x)$ converges.

We first define the $n-$th error term
\begin{equation}
E_n(x) := f(x) - P_n(x)
\end{equation}
and expand out $P_n(x)$.
\begin{equation}
E_n(x) = f(x) - f(a) - f'(a)\cdot (x - a) - \dots - \dfrac{f^{(n)}(a)}{n!}\cdot (x - a)^n
\end{equation}
By $(1)$, $P_n(x) \to f(x)$ if and only if $E_n(x) \to 0$. Look at equation $(2)$, and recall Taylor's Theorem. These look eerily familiar.
\begin{SNP}{\thm}(Taylor-Lagrange) Let $f$ be infinitely differentiable in $(a - h, a + h)$ for some $h > 0$. Then for every $n \in \bb N$ and for every $a\neq x \in (a - h, a + h)$, there is some $\xi$ in $(x, a)$ or $(a, x)$ whichever the case may be such that
$$
E_n(x) = \dfrac{f^{(n + 1)}(\xi)}{(n + 1)!}(x - a)^{n + 1}
$$ 
\end{SNP}
\newpage
This is precisely the statement of Taylor's Theorem. We also have the following which follows from equation $(1)$.
\begin{SNP}{\thm}Everything as above, $P_n(x) \to f(x)$ if and only if $E_n(x) \to 0$.
\end{SNP}
Let's look at some examples.
\begin{SNP}{\xmp}Let $f(x) = \frac{1}{1 - x}$, verify that $f^{(n)}(x)$ exists on $(-1, 1)$ for every $n$ and is given by
$$
f^{(n)}(x) = \dfrac{n!}{(1 - x)^{n + 1}}
$$
Therefore the derivatives at $0$ are
$$
f^{(n)}(0) = n!
$$
and hence the Taylor Series around $0$ is given by
$$
\sum \dfrac{f^{(n)}(0)}{n!}\cdot x^n = \sum x^n
$$
The error term for the $n-$th partial sum is
$$
E_n(x) = \dfrac{f^{(n + 1)}(\xi)}{(n + 1)!}\cdot x^n = \dfrac{1}{(1 - \xi)^{n + 2}}\cdot x^{n + 1}
$$
which indeed $\to 0$ whenever $\abs{x} < 1$.
\end{SNP}
We can similarly show that the following series converge to their respective functions.
\begin{align*}
\sum \dfrac{x^n}{n!} &\to e^x\\
\sum (-1)^n\dfrac{x^{2n + 1}}{(2n + 1)!} &\to \sin x\\
\sum (-1)^{n + 1}\dfrac{x^n}{n} &\to \ln (1 + x)
\end{align*}
and so on.

Having continuous $n-$th derivatives for all $n$ is not sufficient for convergence. Consider
$$
f(x) = \begin{cases}
e^{-\frac{1}{x^2}} & x > 0\\
0 & x\leq 0
\end{cases}
$$
Then $f^{(k)}(0) = 0$ for every $k$ and thus the Taylor Series is identically $0$. Clearly it doesn't converge to $f(x)$ for any $x > 0$, since $f(x) \neq 0$ here.
\end{document}