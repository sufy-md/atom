\documentclass[a4paper, 14pt]{extarticle}

\usepackage{amsmath}
\usepackage{amssymb}
\usepackage{amsthm}
\usepackage{commath}
\usepackage[margin=0.5in]{geometry}
% \usepackage{lexend}
\usepackage{microtype}
\usepackage{parskip}
\usepackage{tikz}
\usepackage{tikz-cd}
\usepackage{tkz-euclide}
\usepackage{xparse}

\usetikzlibrary{calc,angles,quotes, positioning, shapes.geometric}

\theoremstyle{definition}
\newtheorem{dfn}{Definition}
\newtheorem{clm}{Claim}
\newtheorem{asn}{Assertion}
\newtheorem{thm}{Theorem}
\newtheorem{prb}{Problem}
\newtheorem{ans}{Answer}
\newtheorem{lm}{Lemma}
\newtheorem{rmk}{Remark}
\newtheorem{crl}{Corollary}
\newtheorem{ex}{Exercise}
\newtheorem{xmp}{Example}

\newcommand{\titleheader}[1]{\begin{centering}
\begin{LARGE}
\textbf{#1}
\end{LARGE}\\
\hrulefill

\vspace{-1.25\baselineskip}
\hrulefill
\end{centering}\\\\}

\def\changemargin#1#2{\list{}{\rightmargin#2\leftmargin#1}\item[]}
\let\endchangemargin=\endlist

\NewDocumentEnvironment{smrg}{}{\begin{changemargin}{0.5cm}{0.5cm}}{\end{changemargin}}

\NewDocumentEnvironment{SWP}{m m}
{%
  \vspace{-.9cm}%
  \begin{changemargin}{0.5cm}{0.5cm}%
  \noindent#1~#2
  \par
  \textbf{Proof.} 
}
{%
  \qed
  \end{changemargin}
}

\NewDocumentEnvironment{SNP}{m}
{%
  \vspace{-.9cm}%
  \begin{changemargin}{0.5cm}{0.5cm}%
  \noindent#1
}
{%
  \end{changemargin}
}

\newcommand{\bb}[1]{\mathbb{#1}}
\newcommand{\st}{\space \mid \space}
\newcommand{\union}[1]{\displaystyle\mathop{\cup}\limits_{#1}}
\newcommand{\intersect}[1]{\displaystyle\mathop{\cap}\limits_{#1}}
\newcommand{\paran}[1]{\left ( {#1} \right )}
\newcommand{\contra}{$\rightarrow\!\leftarrow$}
\newcommand{\kvec}[2]{({#1}_1 \dots {#1}_{#2})}
\newcommand{\pf}{\textbf{Proof.} }

\newcommand{\AnswerSection}{
    \newpage
    \section*{Answers to Exercises}
    \textit{The following are brief solutions or hints. You are encouraged to review the exercises before checking the answers.}
}
\begin{document}
\titleheader{Series II}
\textbf{Prereqs} RA-07, RA-18

We are now ready to explore the various sufiicient conditions for a series to converge.

Recall from RA-07 that every bounded monotone sequence converges. Let $\{a_n\}_n$ be a sequence such that $a_n > 0$ for every $n$. Then, $S_n$ is a monotone increasing sequence. This insight leads us to the following result, which is a building block for further results.

\begin{SWP}{\thm}{Let $a_n > 0$ for every $n$. Then $\sum a_n$ converges if and only if the sequence $S_n$ is bounded.}Since $a_n > 0$, $S_n$ is monotone increasing. Thus if it is bounded it converges. On the other hand, every convergent sequence is bounded, so if $S_n$ converges it is bounded.
\end{SWP}

This seemingly obvious (and even unnecessary) result directly helps us prove the following test.

\begin{SWP}{\thm}{(Comparison Test) Let $0 \leq a_n \leq b_n$ for every $n \geq k$ for some $k$. If $\sum b_n$ converges, then $\sum a_n$ converges. If $\sum a_n$ diverges, then $\sum b_n$ diverges.}If $\sum b_n$ converges then the sequence of its partial sums (say $T_n$) is bounded. Since $0 \leq a_n \leq b_n$, we get $0 \leq S_n \leq T_n$ and thus $S_n$ is bounded. Thus $\sum a_n$ converges.

Now suppose $\sum a_n$ diverges, i.e. for every $M > 0$, $S_n$ is eventually greater than $M$. Since $0 \leq S_n \leq T_n$, $T_n$ is also unbounded. Thus $\sum b_n$ diverges.
\end{SWP}

If a (positive) series is bounded between $0$ and a (positive) convergent series, it must converge. If a (positive) series is bounded below by a (positive) divergent series, it must diverge. These are statements which are not only intuitively true, but are supported by our definitions of series and convergence.

The existence of comparison test directly gives rise to the following.
\begin{SNP}{\thm}{(Limit Comparison Test) Let $a_n, b_n \geq 0$ for every $n \geq k$ for some $k$. Suppose $\dfrac {a_n}{b_n} \to L$. Then
\begin{enumerate}
	\item If $L \neq 0$ and $L \neq \infty$, then $a_n$ and $b_n$ converge and diverge together.
	\item If $L = 0$ and $b_n$ converges, then $a_n$ converges.
	\item If $L = \infty$ and $b_n$ diverges, then $a_n$ diverges.
\end{enumerate}}For every $\epsilon$ and every $n \geq N$ for some $N$, $a_n$ lies within $(b_n(L - \epsilon), b_n(L + \epsilon))$.
\end{SNP}
The idea is simple. If $\frac{a_n}{b_n} \to L$ and $L$ is neither $0$ nor $\infty$, then they are ``comparable'' and should behave the same. If $L = 0$, then $a_n$ is ``much smaller'' than $b_n$, so $b_n$ converging naturally implies $a_n$ converging. Finally if $L = \infty$, then $a_n$ is ``much larger'' than $b_n$, and so $b_n$ diverging natually implies $a_n$ diverging.
\begin{smrg}\textbf{Proof.} Since $a_n, b_n$ are both $\geq 0$ by assumption, then $\sum a_n$ and $\sum b_n$ converge if and only if they are bounded. In other words, if they don't converge, they necessarily go to $\infty$.

Suppose $\sum b_n$ converges to some $B$ and take $\epsilon = \min\left\{B, \dfrac 1 B\right\}$. Then there is some $M$ such that $\sum\limits_{m = 1}^{n}b_m$ lies within $\epsilon$ of $B$.

Consider the $k-$th partial sum of $\sum a_n$ with $k > \max\{M, N\}$. Then,
$$
0 \leq \sum\limits_{m = 1}^{k}a_m \leq \sum\limits_{m = 1}^{k}b_m(L + \epsilon) \leq B(L + \epsilon) < BL + 1
$$
Where the first inequality is by hypothesis, the second is from $\dfrac{a_n}{b_n} \to L$, the third is from the fact that a partial sum of positive terms is smaller than the complete sum, and the fourth from the fact that $\epsilon \leq \dfrac 1 B$.

All in all, we obtain that all partial sums of $\sum a_n$ are bounded. Combined with $a_n \geq 0$ we get that $\sum a_n$ converges.

Now suppose $\sum b_n$ diverges. Then for every $M > 0$, there is some $N > 0$ such that the $k-$th partial sum of $\sum b_n$ for every $k > N$ is larger than $M$.

Take $M' > 0$, we want to show a similar statement for $\sum a_n$ holds. This is simple, choose $\epsilon$ as before, and use $a_n$ lies within $(b_n(L - \epsilon), b_n(L + \epsilon))$ for every $n \geq N'$ for some $N'$.

Since $b_n$ is unbounded, so is $b_n(L - \epsilon)$, and thus so is $a_n$. Therefore $\sum a_n$ diverges.\qed
\end{smrg}
The proofs for the remaining two cases follow similarly. There are 2 ``missing'' cases, $L = 0$ and $\sum b_n$ diverges, and $L = \infty$ and $\sum b_n$ converges.

In the case $L = 0$ and $\sum b_n$ diverging, following the second part of the proof, we will obtain $\sum a_n$ is bounded below by $0$, which does not help us conclude anything.

Similarly, in the case $L = \infty$ and $\sum b_n$ converging, following the first part of the proof, we will get $\sum a_n$ is bounded above by $\infty$ which doesn't help us conclude anything.

It is now about time we revisit an old friend.
\begin{SNP}{\thm}{(Cauchy Condensation Test) Let $a_n$ be monotone decreasing and $a_n \geq 0$ for every $n$. Then, $\sum a_n$ converges iff $\sum 2^k\cdot a_{2^k}$ converges.}
\end{SNP}
This theorem says that if $a_n$ is monotone decreasing, then
$$
a_1 + a_2 + a_3 + \ldots
$$
converges if and only if
$$
a_1 + 2a_2 + 4a_4 + \ldots
$$
converges. The idea behind the proof is more or less similar to how we showed $\sum \dfrac 1 n$ diverges in RA-18.
\begin{smrg}\textbf{Proof.} Let $S_n$ denote $a_1 + a_2 + \dots + a_n$ and $T_k$ denote $a_1 + 2a_2 + \dots + 2^ka_{2^k}$. Suppose $T_k$ converges. For any $n$, take $k$ such that $2^k \geq n$, then
\begin{align*}
S_n &= a_1 + a_2 + \dots + a_n\\
    &\leq a_1 + a_2 + \dots a_{2^k} + \dots + a_{2^{k + 1} - 1}\\
    &= a_1 + (a_2 + a_3) + (a_4 + \dots a_7) + \dots + (a_{2^k} + \dots + a_{2^{k + 1} - 1})\\
    &\leq a_1 + 2a_2 + \dots + 2^ka_{2^k}\\
    &= T_k
\end{align*}
Therefore $S_n$ is bounded hence converges. Similarly suppose $S_n$ converges, and for some $k$ choose $n$ such that $n \geq 2^k$. Then,
\begin{align*}
S_n &= a_1 + a_2 + \dots + a_n\\
    &\geq a_1 + a_2 + (a_3 + a_4) + (a_5 + \dots + a_8) + \dots + (a_{2^{k - 1} + 1} + \dots a_{2^k})\\
    &\geq \dfrac1 2 a_1 + a_2 + 2a_4 + 4a_8 + \dots + 2^{k-1}a_{2^k}\\
    &= \dfrac 1 2 T_k
\end{align*}
Therefore $T_k$ is bounded hence converges.\qed
\end{smrg}
The Condensation Test is useful in classifying the following class of series.
\begin{SNP}{\xmp}$\sum \frac{1}{n^p}$ converges iff $\sum \frac{2^k}{2^{kp}} = \sum \frac{1}{2^{k(p - 1)}}$ converges. This is a geometric series with common ratio $2^{1 - p}$, thus it converges iff $2^{1 - p} < 1$, or equivalently iff $p > 1$.
\end{SNP}
This isn't the only example where a bound above by a geometric series is useful.
\begin{SWP}{\thm}{(Ratio Test) Let $a_n \neq 0$. Then,
\begin{enumerate}
	\item If $r_n := \abs{\dfrac{a_{n + 1}}{a_n}} \leq q < 1$ for every $n \geq N$ for some $N$, then $\sum a_n$ converges.
	\item If $r_n \geq 1$ for every $n \geq N$ for some $N$, then $\sum a_n$ does not converge.
\end{enumerate}
}\begin{enumerate}\item It is trivial to see that in this case, $\abs {a_{N + p}} \leq q^p \abs{a_N}$, hence $\sum \abs{a_n}$ is bounded above by a convergent geometric series.
\item In this case, $a_n \not \to 0$.\end{enumerate}As required.\end{SWP}Observe that $\lim\limits_{n\to \infty} \abs{\frac{a_{n + 1}}{a_n}}$ need not exist for us to apply the ratio test. Just that the ratio should eventually be bounded above by some $q < 1$, or bounded below by $1$. However, in the case the limit exists, we can make similar statements.
\begin{SNP}{\crl}Let $L = \lim\limits_{n\to\infty}\abs{\frac{a_{n + 1}}{a_n}}$. Then,
\begin{enumerate}
	\item If $L > 1$, then $\sum a_n$ does not converge.
	\item If $L < 1$, then $\sum a_n$ converges.
	\item If $L = 1$, we cannot conclude.
\end{enumerate}
\end{SNP}
Statements 1 and 2 more or less directly follow from the previous theorem. For statement 3, consider the series $\sum \frac 1 n$ and $\sum \frac {1} {n^2}$. The former doesn't converge, the latter converges, and the ratio converges to 1 for both of them.

There is another way to bound a series by a geometric series.
\begin{SWP}{\thm}{(Root Test) If $r_n := \abs{a_n}^{\frac1 n} \leq q < 1$ for every $n \geq N$ for some $N$, then $\sum a_n$ converges.}Again, this means $\abs{a_n} \leq q^n$ for every $n \geq N$ thus bounding the series above by a geometric one.
\end{SWP}
This test also has its limit analog.
\begin{SNP}{\crl}Suppose $\lim\limits_{n\to\infty} = L$. If $L < 1$, then $\sum a_n$ converges. If $L > 1$, then $\sum a_n$ does not converge. If $L = 1$, we cannot make any conclusions.
\end{SNP}
In both the ratio and root tests, we have implicitly assumed the following.
\begin{SNP}{\ex}{Suppose $\sum \abs{a_n}$ converges. Then show that $\sum a_n$ converges. (Hint. A sequence converges iff it is Cauchy.)}
\end{SNP}
Finally we address the alternating series test.
\begin{SWP}{\thm}{(Alternating Series / Leibniz Test) If $a_n \geq 0$ is monotone decreasing and $a_n \to 0$, then $\sum (-1)^{n + 1}a_n$, or
$$
a_1 - a_2 + a_3 - a_4 + \dots
$$converges.}Observe that the quantities $a_1 - a_2, a_2 - a_3$ or $a_k - a_{k + 1}$ in general are all $\geq 0$. This will be useful.

First consider the sum ending at a general even term
\begin{align*}
S_{2k} &= a_1 - a_2 + a_3 - a_4 + ... + a_{2k - 1} - a_{2k}\\
       &\leq a_1 - a_2 + ... + a_{2k - 1}\\
       &= a_1 - (a_2 - a_3) - (a_4 - a_5) - \dots - (a_{2k - 2} - a_{2k - 1})\\
       &\leq a_1
\end{align*}
Further,
$$
S_{2(k + 1)} = S_{2k + 2} = S_{2k} + (a_{2k + 1} - a_{2k + 2}) \geq S_{2k}
$$
Thus $S_{2k}$ is increasing and bounded above by $S_1$. Thus $S_{2k}$ converges.

Similarly consider the sum ending at a general odd term.
\begin{align*}
S_{2k + 1} &= a_1 - a_2 + a_3 - a_4 + \dots -a_{2k} + a_{2k + 1}\\
           &\geq a_1 - a_2 + \dots + a_{2k - 1} - a_{2k}\\
           &= (a_1 - a_2) + (a_3 - a_4) + (a_5 - a_6) + \dots (a_{2k - 1} - a_{2k})\\
           &\geq (a_1 - a_2)
\end{align*}
Further,
$$
S_{2(k + 1) + 1} = S_{2k + 3} = S_{2k + 1} - (a_{2k + 2} - a_{2k + 3}) \leq S_{2k + 1}
$$
Thus $S_{2k + 1}$ is decreasing and bounded below by $S_2$. Thus $S_{2k + 1}$ converges.

i.e., the sequences of partial sums upto odd and upto even terms both converge. Since $S_{2n + 1} - S_{2n} = a_{2n + 1} \to 0$, they both converge to the same limit.

Therefore $S_n$ converges.
\end{SWP}
There's a subtle point to be noted here. If $S_{2n + 1} - S_{2n}$ does not go to $0$, then necessarily $a_n \not \to 0$ and the sum doesn't converge. Therefore,
\begin{SNP}{\crl}Let $a_n$ be monotone decreasing with $a_n \geq 0$. Then $\sum (-1)^na_n$ converges if and only if $a_n \to 0$.
\end{SNP}
Of course, there is a lot more to say about series and sufficient conditions for convergence, but this will do for now.
\AnswerSection
\ans Recall a series $\sum a_n$ converges if and only if the sequence of its partial sums $S_n = \sum\limits_{k = 1}^{n}a_k$ converges. Further recall a sequence converges iff it is Cauchy.

Suppose $\sum \abs{a_n}$ converges, thus the sequence $S_n := \sum\limits_{k = 1}^{n}\abs{a_k}$ is Cauchy. Consider the sequence $T_n = \sum\limits_{k = 1}^{n}a_k$. It suffices to show $T_n$ is Cauchy.

Let $\epsilon > 0$. Since $S_n$ is Cauchy, there is some $N \in \bb N$  such that for every $m, n \geq N$, $\abs{S_m - S_n} < \epsilon$.

Now consider $\abs{T_m - T_n}$ for $m, n \geq N$, with $n \geq m$.
\begin{align*}
\abs{T_n - T_m} &= \abs{\sum\limits_{k = m + 1}^{n}a_k}\\
                &\leq \sum\limits_{k = m + 1}^{n}\abs{a_k}\\
                &= S_n - S_m\\
                &= \abs{S_n - S_m}\\
                &< \epsilon
\end{align*}
Therefore $T_n$ is Cauchy and thus converges.
\end{document}