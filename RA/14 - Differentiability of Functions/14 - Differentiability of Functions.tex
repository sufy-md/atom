\documentclass[a4paper, 14pt]{extarticle}

\usepackage{amsmath}
\usepackage{amssymb}
\usepackage{amsthm}
\usepackage{commath}
\usepackage[margin=0.5in]{geometry}
% \usepackage{lexend}
\usepackage{microtype}
\usepackage{parskip}
\usepackage{tikz}
\usepackage{tikz-cd}
\usepackage{tkz-euclide}
\usepackage{xparse}

\usetikzlibrary{calc,angles,quotes, positioning, shapes.geometric}

\theoremstyle{definition}
\newtheorem{dfn}{Definition}
\newtheorem{clm}{Claim}
\newtheorem{asn}{Assertion}
\newtheorem{thm}{Theorem}
\newtheorem{prb}{Problem}
\newtheorem{ans}{Answer}
\newtheorem{lm}{Lemma}
\newtheorem{rmk}{Remark}
\newtheorem{crl}{Corollary}
\newtheorem{ex}{Exercise}
\newtheorem{xmp}{Example}

\newcommand{\titleheader}[1]{\begin{centering}
\begin{LARGE}
\textbf{#1}
\end{LARGE}\\
\hrulefill

\vspace{-1.25\baselineskip}
\hrulefill
\end{centering}\\\\}

\def\changemargin#1#2{\list{}{\rightmargin#2\leftmargin#1}\item[]}
\let\endchangemargin=\endlist

\NewDocumentEnvironment{smrg}{}{\begin{changemargin}{0.5cm}{0.5cm}}{\end{changemargin}}

\NewDocumentEnvironment{SWP}{m m}
{%
  \vspace{-.9cm}%
  \begin{changemargin}{0.5cm}{0.5cm}%
  \noindent#1~#2
  \par
  \textbf{Proof.} 
}
{%
  \qed
  \end{changemargin}
}

\NewDocumentEnvironment{SNP}{m}
{%
  \vspace{-.9cm}%
  \begin{changemargin}{0.5cm}{0.5cm}%
  \noindent#1
}
{%
  \end{changemargin}
}

\newcommand{\bb}[1]{\mathbb{#1}}
\newcommand{\st}{\space \mid \space}
\newcommand{\union}[1]{\displaystyle\mathop{\cup}\limits_{#1}}
\newcommand{\intersect}[1]{\displaystyle\mathop{\cap}\limits_{#1}}
\newcommand{\paran}[1]{\left ( {#1} \right )}
\newcommand{\contra}{$\rightarrow\!\leftarrow$}
\newcommand{\kvec}[2]{({#1}_1 \dots {#1}_{#2})}
\newcommand{\pf}{\textbf{Proof.} }

\newcommand{\AnswerSection}{
    \newpage
    \section*{Answers to Exercises}
    \textit{The following are brief solutions or hints. You are encouraged to review the exercises before checking the answers.}
}
\begin{document}
\titleheader{Differentiability of Functions}
\textbf{Prereqs} FD-02, RA-11

So far, we've dealt with whether you can draw a function without lifting your pen. Now we ask -- can you draw it without snapping your wrist?

The idea of \emph{differentiability} deals with this question, and is the central idea in understanding velocity, curvature, and continuous optimization. (Differentiability has a \emph{much} different meaning in a several variable context, but is more or less a generalization. Weird how that works, right?).

Consider the graph of $f(x) = \abs{x}$
\begin{center}
\begin{tikzpicture}[x=2cm, y=2cm]
  % Axes
  \draw[->] (-2.5, 0) -- (2.5, 0) node[right] {$x$};
  \draw[->] (0, -0.5) -- (0, 2.5) node[above] {$y$};

  % Plot y = |x|
  \draw[thick, blue, domain=-2:0] plot (\x, {-\x});
  \draw[thick, blue, domain=0:2]  plot (\x, {\x});

  % Label origin
  \node[below left] at (0,0) {$0$};
\end{tikzpicture}\\
The pen does not lift at $(0, 0)$, but your wrist will snap.
\end{center}
You absolutely can draw this without lifting your pen off the paper, but you'd have to sharply change directions at the origin. Compare that to the curve of $f(x) = x^2$
\begin{center}
\begin{tikzpicture}[x=2cm, y=2cm]
  % Axes
  \draw[->] (-1.5, 0) -- (1.5, 0) node[right] {$x$};
  \draw[->] (0, -0.5) -- (0, 2.75) node[above] {$y$};

  % Graph of y = x^2
  \draw[domain=-1.5:1.5, smooth, variable=\x, thick, blue] plot ({\x}, {\x*\x});

  % Notable points
  \filldraw (0,0) circle (1pt) node[below left] {$(0,0)$};
\end{tikzpicture}\\
The pen glides smoothly from its downward movement to upward, like butter.
\end{center}
\newpage
Differentiability, like continuity, is defined at a point. For example, your pen \emph{will} move smoothly when drawing $f(x) = \abs x$ at all points other than the origin. So it becomes unfair to say that $f$ is not differentiable, when only one point serves as counterexample.

So how do we quantify this lack of ``jerkiness''? We want to capture that at a given point $(x_0, f(x_0))$, the movement of the pen is \emph{smooth}. The way we do this is simple. If the direction in which your pen approaches a point $(x_0 , f(x_0))$ is the same as the direction in which it exits, then we consider the movement to be smooth.

Contrast this against $\abs{x}$ at $(0, 0)$. We enter in the direction $(1, -1)$ and exit in the direction $(1, 1)$. This idea is thus intuitively strong enough to form the definition.

\begin{center}
\begin{tikzpicture}[x=2cm, y=2cm]
  % Axes
  \draw[->] (-2.5, 0) -- (2.5, 0) node[right] {$x$};
  \draw[->] (0, -0.5) -- (0, 2.5) node[above] {$y$};

  % Plot y = |x|
  \draw[thick, blue, domain=-2:0] plot (\x, {-\x});
  \draw[thick, blue, domain=0:2]  plot (\x, {\x});

  % Label origin
  \node[below left] at (0,0) {$0$};

  % Direction arrows with vector labels
  \draw[->, red, thick] (-1.2,1.2) -- (-0.4,0.4);
  \node[left] at (-1.2,1.2) {\footnotesize $(1,\,-1)$};

  \draw[->, red, thick] (0.4,0.4) -- (1.2,1.2);
  \node[right] at (1.2,1.2) {\footnotesize $(1,\,1)$};

  \draw[->, red, thick] (-1.2,1.2) -- (-0.4,0.4) node[pos=1, left] {\footnotesize slope $-1$};
  \draw[->, red, thick] (0.4,0.4) -- (1.2,1.2) node[pos=0, right] {\footnotesize slope $+1$};
\end{tikzpicture}\\
We enter in the direction of the vector $(1, -1)$ and exit in $(1, 1)$
\end{center}

\begin{SNP}{\dfn}{A function $f: \Omega \rightarrow \bb R$ with $\Omega \subseteq \bb R$ is \emph{differentiable at} $x_0$ if the limits
$$
\lim_{h\rightarrow0^+} \frac{f(x_0 + h) - f(x_0)}{h}
$$
and
$$
\lim_{h\rightarrow0^+} \frac{f(x_0) - f(x_0 - h)}{h}
$$
exist and are equal.}
\end{SNP}
Where to quantify direction of exiting (resp. entering) the point, we use the slope of the line between our concerned point and a point slightly ahead (resp. behind) it.

Note that the second limit in the definition is precisely equal to the following limit.
$$
\lim_{h\rightarrow0^-}\frac{f(x_0 + h) - f(x_0)}{h}
$$

This equality helps us frame a second definition.
\begin{SNP}{\dfn}{A function $f: \Omega \rightarrow \bb R$ with $\Omega \subseteq \bb R$ is \emph{differentiable at} $x_0$ if the limit
$$
\lim_{h\rightarrow0} \frac{f(x_0 + h) - f(x_0)}{h}
$$exists.}
\end{SNP}

Of course, we need to justify the restatement.

\begin{smrg}
\textbf{Proof.} (Defn. 1 $\implies$ Defn. 2) Suppose for some $f$ both limits in Defn. 1 exist and are equal at some $x_0$. Consider the limit in Defn. 2. Take any sequence $h_n \rightarrow 0$ with $h_n \neq 0$ for every $n$ and divide it into subsequences $h_{n_k}$ and $h_{n_{k'}}$ with $h_{n_k} > 0$ and $h_{n_{k'}} < 0$. Then by Defn. 1, the sequences 
$$
\frac{f(x_0 + h_{n_k}) - f(x_0)}{h_{n_k}}
$$
and
$$
\frac{f(x_0 + h_{n_{k'}}) - f(x_0)}{h_{n_{k'}}}
$$
converge and are equal to $L$ (say). Apply the definition of convergence to both these sequences for some $\epsilon$ to obtain $N_1, N_2$.

Now consider the sequence
$$
\frac{f(x_0 + h_n) - f(x_0)}{h_n}
$$
Whenever $n \geq \max\{N_1, N_2\}$, by convergence of above two sequences (since $h_n$ is either $>0$ or $<0$, it falls into precisely one of $h_{n_k}$ and $h_{n_{k'}}$) we get that
$$
\abs{\frac{f(x_0 + h_n) - f(x_0)}{h_n} - L} < \epsilon
$$
Therefore we obtain
$$
\lim_{h\rightarrow 0}\frac{f(x_0 + h) - f(x)}{h} = L
$$
as required.\qed

(Defn. 2 $\implies$ Defn. 1) We know for any sequence $h_n \rightarrow 0$, the limit
$$
\lim_{h_n\rightarrow 0}\frac{f(x_0 + h_n) - f(x_0)}{h_n} = L
$$
Since $h_n \rightarrow 0^+$ and $h_n \rightarrow 0^-$ both imply $h_n \rightarrow 0$, their respective limits of the difference quotient also converge to $L$.\qed
\end{smrg}
Whenever a function $f$ is differentiable at $x_0$, the limit $L$ of the difference quotient is usually denoted by $f'(x_0)$.

If $f'$ is defined for all $x_0 \in \Omega$, we say $f$ is \emph{differentiable on} $\Omega$ and that $f'$ is the \emph{derivative of} $f$.

Suppose now there is some curve which we can draw smoothly on a set $\Omega$. Does it intuitively make sense that we can draw it without lifting our pen? Of course! Semi-formally, take the contrapositive. If I have to lift my pen to draw it at some point $x_0$, how can I smoothly draw it at $x_0$? I can't!

\begin{SWP}{\thm}{If $f: \Omega \rightarrow \bb R$ with $\Omega \subseteq \bb R$ is differentiable at some $x_0 \in \Omega$, then $f$ is continuous at $x_0$.}Suppose $f$ is differentiable at $x_0$. Take any sequence $x_n \rightarrow x_0$ with $x_n \neq x_0$ for every $n$. Then the limit
$$
\lim_{n\rightarrow \infty}\frac{f(x_n) - f(x_0)}{x_n - x_0}
$$
exists by definition of differentiability, and is denoted by $f'(x_0)$. Further, we know that $x_n - x_0 \rightarrow 0$. Now observe that
\begin{align}
\lim_{n \rightarrow \infty} f(x_n) - f(x_0) &= \lim_{n \rightarrow \infty} (x_n - x_0)\cdot\frac{f(x_n) - f(x_0)}{x_n - x_0}\\
                                            &= \lim_{n \rightarrow \infty} (x_n - x_0)\cdot\lim_{n \rightarrow \infty}\frac{f(x_n) - f(x_0)}{x_n - x_0}\\
                                            &= 0\cdot f'(x_0)\\
                                            &= 0
\end{align}
Where $(1)$ is valid because $x_n - x_0$ is never $0$ by choice of $x_n$, and we split the limit in $(1)$ into two limits in $(2)$ because both limits exist.

Therefore $f(x_n) \rightarrow f(x_0)$ and $f$ is continuous.
\end{SWP}

So functions differentiable on some interval $(a, b)$ are also continuous on $(a, b)$. Hence, functions differentiable on $(a, b)$ and continuous on $[a, b]$ have some interesting properties -- including the Mean Value Theorems and characterizations of local extrema which we will now look at.

Observe the hesitation in calling a function differentiable on $[a, b]$. We cannot have a function $f$ defined on $[a, b]$ differentiable on $a$, because the second limit in the original definition (or the ``entry slope'') will not be defined. Similarly, we cannot have $f$ differentiable on $b$, because the first limit (or the ``exit slope'') will not be defined.
\end{document}