\documentclass[a4paper, 14pt]{extarticle}

\usepackage{amsmath}
\usepackage{amssymb}
\usepackage{amsthm}
\usepackage{commath}
\usepackage[margin=0.5in]{geometry}
% \usepackage{lexend}
\usepackage{microtype}
\usepackage{parskip}
\usepackage{tikz}
\usepackage{tikz-cd}
\usepackage{tkz-euclide}
\usepackage{xparse}

\usetikzlibrary{calc,angles,quotes, positioning, shapes.geometric}

\theoremstyle{definition}
\newtheorem{dfn}{Definition}
\newtheorem{clm}{Claim}
\newtheorem{asn}{Assertion}
\newtheorem{thm}{Theorem}
\newtheorem{prb}{Problem}
\newtheorem{ans}{Answer}
\newtheorem{lm}{Lemma}
\newtheorem{rmk}{Remark}
\newtheorem{crl}{Corollary}
\newtheorem{ex}{Exercise}
\newtheorem{xmp}{Example}

\newcommand{\titleheader}[1]{\begin{centering}
\begin{LARGE}
\textbf{#1}
\end{LARGE}\\
\hrulefill

\vspace{-1.25\baselineskip}
\hrulefill
\end{centering}\\\\}

\def\changemargin#1#2{\list{}{\rightmargin#2\leftmargin#1}\item[]}
\let\endchangemargin=\endlist

\NewDocumentEnvironment{smrg}{}{\begin{changemargin}{0.5cm}{0.5cm}}{\end{changemargin}}

\NewDocumentEnvironment{SWP}{m m}
{%
  \vspace{-.9cm}%
  \begin{changemargin}{0.5cm}{0.5cm}%
  \noindent#1~#2
  \par
  \textbf{Proof.} 
}
{%
  \qed
  \end{changemargin}
}

\NewDocumentEnvironment{SNP}{m}
{%
  \vspace{-.9cm}%
  \begin{changemargin}{0.5cm}{0.5cm}%
  \noindent#1
}
{%
  \end{changemargin}
}

\newcommand{\bb}[1]{\mathbb{#1}}
\newcommand{\st}{\space \mid \space}
\newcommand{\union}[1]{\displaystyle\mathop{\cup}\limits_{#1}}
\newcommand{\intersect}[1]{\displaystyle\mathop{\cap}\limits_{#1}}
\newcommand{\paran}[1]{\left ( {#1} \right )}
\newcommand{\contra}{$\rightarrow\!\leftarrow$}
\newcommand{\kvec}[2]{({#1}_1 \dots {#1}_{#2})}
\newcommand{\pf}{\textbf{Proof.} }

\newcommand{\AnswerSection}{
    \newpage
    \section*{Answers to Exercises}
    \textit{The following are brief solutions or hints. You are encouraged to review the exercises before checking the answers.}
}
\begin{document}
\titleheader{Real Valued Sequences - I}
\textbf{Prereqs} INT-01 

When we refer to real-valued sequences, we usually mean sequences indexed by $\bb N$. Therefore, specifically for real analysis, the following definition is standard.
\begin{SNP}{\dfn}{A \emph{sequence} is a function $f : \bb N \rightarrow \bb R$}
\end{SNP}

A favorite example is the arithmetic sequences $$f(n) = a + (n - 1)d$$ whose terms are usually denoted by $a_n$. Observe that it is not necessary that the function $f$ can be written in a nice clean form. As long as it is defined, $f$ forms a sequence. For example, consider the sequence
$$
1, 6, 1, 8, 0, 3, 3, 9, 8, 8, 7, 4, 9 \dots
$$
Do you see a pattern? Maybe you do recognise what the sequence is, but I assure you, there is no \emph{pattern} per se. Regardless, the sequence can still be listed via a function $f: \bb N \rightarrow \bb R$ if we just let
$$
f(1) = 1, f(2) = 6, f(3) = 1 \dots
$$
and so on. Any list of real numbers defines a real-valued sequence.

How about sequence of a sequence? say I have an arithmetic progression
$$
1, 4, 7, 10, 13, 16, 19 \dots
$$
Let these be $a_1, a_2, a_3 \dots$ respectively. If we then take only the even-indexed terms $a_2, a_4 \dots$ we get the sequence
$$
4, 10, 16 \dots
$$
which is again an arithmetic progression and therefore a sequence. Here, since the second sequence is taken from the first, we call it a \emph{subsequence}.

\begin{SNP}{\dfn}
{
	Let $f: \bb N \rightarrow \bb R$ be a sequence, and $g: \bb N \rightarrow \bb N$ a monotone increasing function. Then, $f \circ g : \bb N \rightarrow \bb R$ is a \emph{subsequence} of $f$
}
\end{SNP}
In the above example, $f(n) = 3n -2$ which describes the sequence, and $g(n) = 2n$ which describes the \emph{sequence of indices of $f$ we want to take}. Indeed, $$f \circ g(n) = f(2n) = 6n - 2$$ which describes the subsequence.

\end{document}