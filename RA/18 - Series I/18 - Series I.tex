\documentclass[a4paper, 14pt]{extarticle}

\usepackage{amsmath}
\usepackage{amssymb}
\usepackage{amsthm}
\usepackage{commath}
\usepackage[margin=0.5in]{geometry}
% \usepackage{lexend}
\usepackage{microtype}
\usepackage{parskip}
\usepackage{tikz}
\usepackage{tikz-cd}
\usepackage{tkz-euclide}
\usepackage{xparse}

\usetikzlibrary{calc,angles,quotes, positioning, shapes.geometric}

\theoremstyle{definition}
\newtheorem{dfn}{Definition}
\newtheorem{clm}{Claim}
\newtheorem{asn}{Assertion}
\newtheorem{thm}{Theorem}
\newtheorem{prb}{Problem}
\newtheorem{ans}{Answer}
\newtheorem{lm}{Lemma}
\newtheorem{rmk}{Remark}
\newtheorem{crl}{Corollary}
\newtheorem{ex}{Exercise}
\newtheorem{xmp}{Example}

\newcommand{\titleheader}[1]{\begin{centering}
\begin{LARGE}
\textbf{#1}
\end{LARGE}\\
\hrulefill

\vspace{-1.25\baselineskip}
\hrulefill
\end{centering}\\\\}

\def\changemargin#1#2{\list{}{\rightmargin#2\leftmargin#1}\item[]}
\let\endchangemargin=\endlist

\NewDocumentEnvironment{smrg}{}{\begin{changemargin}{0.5cm}{0.5cm}}{\end{changemargin}}

\NewDocumentEnvironment{SWP}{m m}
{%
  \vspace{-.9cm}%
  \begin{changemargin}{0.5cm}{0.5cm}%
  \noindent#1~#2
  \par
  \textbf{Proof.} 
}
{%
  \qed
  \end{changemargin}
}

\NewDocumentEnvironment{SNP}{m}
{%
  \vspace{-.9cm}%
  \begin{changemargin}{0.5cm}{0.5cm}%
  \noindent#1
}
{%
  \end{changemargin}
}

\newcommand{\bb}[1]{\mathbb{#1}}
\newcommand{\st}{\space \mid \space}
\newcommand{\union}[1]{\displaystyle\mathop{\cup}\limits_{#1}}
\newcommand{\intersect}[1]{\displaystyle\mathop{\cap}\limits_{#1}}
\newcommand{\paran}[1]{\left ( {#1} \right )}
\newcommand{\contra}{$\rightarrow\!\leftarrow$}
\newcommand{\kvec}[2]{({#1}_1 \dots {#1}_{#2})}
\newcommand{\pf}{\textbf{Proof.} }

\newcommand{\AnswerSection}{
    \newpage
    \section*{Answers to Exercises}
    \textit{The following are brief solutions or hints. You are encouraged to review the exercises before checking the answers.}
}
\begin{document}
\titleheader{Series I}
\textbf{Prereqs} RA-07

A series (or specifically, an infinite series) is essentially a sum of countably many terms. The axioms of $\bb R$ only define a sum of finitely many terms. Through series, we seek to define a sum of infinitely many terms.

Let $a_n$ be a sequence. Then we define a new sequence, $S_n$, given by
$$
S_n = \sum\limits_{m = 1}^{n}a_m
$$
and we define the infinite sum as the limit of $S_n$.
$$
\sum\limits_{m = 1}^{\infty}a_m := \lim_{n\to \infty}S_n = \lim_{n\to \infty}\sum\limits_{m = 1}^{n}a_m
$$
In other words, the infinite sum
$$
a_1 + a_2 + a_3 + \ldots
$$
is defined as the limit of the sequence
$$
a_1, \space a_1 + a_2, \space a_1 + a_2 + a_3 \ldots
$$
which, at first glance, seems to make sense. And it does! If you want to evaluate an infinite sum by hand, this is the exact thing you will do. You'll take $a_1$, and then add $a_2$ (to obtain $S_2$), and then add $a_3$ (to obtain $S_3$) and so on until you get an overall idea of what the sum is ``doing''.

\begin{SNP}{\dfn}A series $\sum a_n$ \emph{converges} if $S_n \to S$ for some $S$. The series \emph{diverges} if $S_n \to \pm \infty$.
\end{SNP}
\begin{SNP}{\xmp}Let $a_n = \dfrac{1}{2^n}$. Then the series $\sum a_n$ converges to $1$. Let $b_n = n$. Then the series $\sum b_n$ diverges. Finally let $c_n = (-1)^n$. Then the series $\sum c_n$ neither converges nor diverges.
\end{SNP}
When can we say that a series converges? Unfortunately, the answer is not straightforward. It is easy to sniff out when a series \emph{doesn't} converge. Suppose $a_n \not \to 0$ as $n \to \infty$. Can $\sum a_n$ converge?

As a handwavy proof, suppose $\sum a_n$ converges, i.e. the sequence $S_n$ converges to some limit $S$. Consider a new sequence defined by $T_n = S_{n + 1}$.

\begin{SNP}{\ex}{Does $T_n$ converge? Where? What is $T_n - S_n$ equal to? Does this converge? Where?}
\end{SNP}
\begin{SNP}{\thm}{If a series $\sum a_n$ converges, then $a_n \to 0$.}
\end{SNP}

This condition is only \textit{necessary}, not sufficient. What that means is, even if $a_n \rightarrow 0$, it may be the case that $S_n$ doesn't converge.

\begin{SNP}{\xmp}Consider $a_n = \dfrac{1}{n}$. Then $\sum a_n$ doesn't converge. Write the $(2^k)$-th partial sums as follows
\begin{align*}
\sum\dfrac1 n&= 1 + \left(\dfrac1 2\right) + \left(\dfrac 1 3 + \dfrac 1 4\right) + \left(\dfrac 1 5 + \dfrac 1 6 + \dfrac 1 7 + \dfrac 1 8\right) + \ldots\\
&\geq 1 + \left(\dfrac1 2\right) + \left(\dfrac 1 4 + \dfrac 1 4\right) + \left(\dfrac1 8 + \dfrac1 8 + \dfrac1 8 + \dfrac1 8\right) + \ldots\\
&= 1 + \dfrac1 2 + \dfrac 1 2 + \dfrac1 2 + \ldots\\
&= 1 + \sum\limits_{n = 1}^{\infty}\dfrac1 2\\
&= \infty
\end{align*}
We will revisit this in the Condensation test and the Rearrangement Theorem.
\end{SNP}
It is in general hard to fully characterise converging series. An important class of examples are the geometric series $a_n = ar^{n - 1}$ with first term $a \neq 0$ and common ratio $r \neq 0$. (The identically $0$ series trivially converge).

\begin{SWP}{\thm}{Let $a_n$ be a geometric series with $a_2 \neq 0$ (neither of $a$ or $r$ is $0$). Then, $\sum a_n$ converges if and only if $\abs{r} < 1$.}$(\rightarrow)$ Suppose $\sum a_n$ converges. Then $a_n \rightarrow 0$, or $ar^{n - 1} \rightarrow 0$. Since $a$ and $r$ are both $\neq 0$, we must have $r^{n - 1} \to 0$ and therefore $\abs{r} < 1$. For if $\abs{r} \geq 1$ then $\abs r^{n - 1} \geq 1$ and in this case $r^{n - 1}$ cannot converge to $0$.

$(\leftarrow)$ Now suppose for contrapositive that $\abs r \geq 1$. As before, in this case $a_n \not \to 0$ and thus $\sum a_n$ doesn't converge.
\end{SWP}

So we have our first test for convergence -- geometric series with ratios whose absolute value is smaller than $1$ converge.

This result is critical in proving the ratio test, a strong sufficient condition for convergence. We will look at that and other tests in RA-19.
\AnswerSection
\ans Yes, $T_n$ also converges to $S$. $T_n - S_n$ is equal to $a_{n + 1}$ by definition. Also, $T_n - S_n$ converges to $S - S = 0$. Thus $a_{n + 1}$ (and thus $a_n$) converges to $0$.
\end{document}