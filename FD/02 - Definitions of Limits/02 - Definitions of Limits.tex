\documentclass[a4paper, 14pt]{extarticle}

\usepackage{amsmath}
\usepackage{amssymb}
\usepackage{amsthm}
\usepackage{commath}
\usepackage[margin=0.5in]{geometry}
% \usepackage{lexend}
\usepackage{microtype}
\usepackage{parskip}
\usepackage{tikz}
\usepackage{tikz-cd}
\usepackage{tkz-euclide}
\usepackage{xparse}

\usetikzlibrary{calc,angles,quotes, positioning, shapes.geometric}

\theoremstyle{definition}
\newtheorem{dfn}{Definition}
\newtheorem{clm}{Claim}
\newtheorem{asn}{Assertion}
\newtheorem{thm}{Theorem}
\newtheorem{prb}{Problem}
\newtheorem{ans}{Answer}
\newtheorem{lm}{Lemma}
\newtheorem{rmk}{Remark}
\newtheorem{crl}{Corollary}
\newtheorem{ex}{Exercise}
\newtheorem{xmp}{Example}

\newcommand{\titleheader}[1]{\begin{centering}
\begin{LARGE}
\textbf{#1}
\end{LARGE}\\
\hrulefill

\vspace{-1.25\baselineskip}
\hrulefill
\end{centering}\\\\}

\def\changemargin#1#2{\list{}{\rightmargin#2\leftmargin#1}\item[]}
\let\endchangemargin=\endlist

\NewDocumentEnvironment{smrg}{}{\begin{changemargin}{0.5cm}{0.5cm}}{\end{changemargin}}

\NewDocumentEnvironment{SWP}{m m}
{%
  \vspace{-.9cm}%
  \begin{changemargin}{0.5cm}{0.5cm}%
  \noindent#1~#2
  \par
  \textbf{Proof.} 
}
{%
  \qed
  \end{changemargin}
}

\NewDocumentEnvironment{SNP}{m}
{%
  \vspace{-.9cm}%
  \begin{changemargin}{0.5cm}{0.5cm}%
  \noindent#1
}
{%
  \end{changemargin}
}

\newcommand{\bb}[1]{\mathbb{#1}}
\newcommand{\st}{\space \mid \space}
\newcommand{\union}[1]{\displaystyle\mathop{\cup}\limits_{#1}}
\newcommand{\intersect}[1]{\displaystyle\mathop{\cap}\limits_{#1}}
\newcommand{\paran}[1]{\left ( {#1} \right )}
\newcommand{\contra}{$\rightarrow\!\leftarrow$}
\newcommand{\kvec}[2]{({#1}_1 \dots {#1}_{#2})}
\newcommand{\pf}{\textbf{Proof.} }

\newcommand{\AnswerSection}{
    \newpage
    \section*{Answers to Exercises}
    \textit{The following are brief solutions or hints. You are encouraged to review the exercises before checking the answers.}
}
\begin{document}
\titleheader{Definitions of Limits}
Here we'll look at the definitions of statements of the form
$$
\lim_{x\rightarrow x_0} f(x) = L
$$
Where $x$ varies over the real numbers (or some subset), and $x_0$ and $L$ are either real numbers or $\pm \infty$

The idea behind the definitions is simple. We'll essentially say ``however close you want $f(x)$ to be to $L$, that is achievable, provided $x$ is closer to $x_0$ than some threshold''.

In the cases where $x_0$ and $L$ are $\infty$, being sufficiently close to $\infty$ is replaced by being sufficiently large. Let's see.

\begin{SNP}{\dfn}{Let $x_0, L \in \bb R$ and $f: \Omega \rightarrow \bb R$ with $\Omega \subseteq \bb R$. Then, we say
$$
\lim_{x\rightarrow x_0} f(x) = L
$$
if for every $\epsilon > 0$, there is some $\delta > 0$ such that whenever
$$
x \in (x_0 - \delta, x_0 + \delta)
$$
then
$$
f(x) \in (L - \epsilon, L + \epsilon)
$$}
\end{SNP}
Now let us see what happens when the independent variable is infinite.
\begin{SNP}{\dfn}{Let $L \in \bb R$ and $f: \Omega \rightarrow \bb R$ with $\Omega \subseteq \bb R$. Then, we say
$$
\lim_{x\rightarrow \infty} f(x) = L
$$
if for every $\epsilon > 0$, there exists $M > 0$ in $\bb R$ such that whenever
$$
x > M
$$
then
$$
f(x) \in (L - \epsilon, L + \epsilon)
$$}
\end{SNP}
Here, $x$ ``getting closer to'' $\infty$ is measured by letting $x$ be larger than an arbitrary number. Of course, $x \rightarrow -\infty$ can be similarly defined.
\begin{SNP}{\dfn}{Let $x_0 \in \bb R$ and $f:\Omega \rightarrow \bb R$ with $\Omega \subseteq \bb R$. Then, we say
$$
\lim_{x\rightarrow x_0} f(x) = \infty
$$
if for every $N > 0$ there is some $\delta > 0$ such that whenever
$$
x \in (x_0 - \delta, x_0 + \delta)
$$
then
$$
f(x) > N
$$}
\end{SNP}
Finally, we state the definition where both variables go to $\infty$.
\begin{SNP}{\dfn}{Let $f:\Omega \rightarrow \bb R$ with $\Omega \subseteq R$. We say
$$
\lim_{x\rightarrow\infty}f(x) = \infty
$$
if for every $N > 0$, there is some $M > 0$ such that whenever
$$
x > M
$$
then
$$
f(x) > N
$$}
\end{SNP}
Observe that every single one of these definitions can be replaced with a sequential counterpart. $x_0, L, $ and $f$ as above, note that the following definitions are respective equivalents of those stated above.
\begin{SNP}{\dfn}{We say that $$\lim_{x\rightarrow x_0}f(x) = L$$ if for every $x_n \rightarrow x_0$, $f(x_n) \rightarrow L$}
\end{SNP}
\begin{SNP}{\dfn}{We say that $$\lim_{x\rightarrow \infty}f(x) = L$$ if for every $x_n \rightarrow \infty$, $f(x_n) \rightarrow L$}
\end{SNP}
\begin{SNP}{\dfn}{We say that $$\lim_{x\rightarrow x_0}f(x) = \infty$$ if for every $x_n \rightarrow x_0$, $f(x_n) \rightarrow \infty$}
\end{SNP}
\begin{SNP}{\dfn}{We say that $$\lim_{x\rightarrow \infty}f(x) = \infty$$ if for every $x_n \rightarrow \infty$, $f(x_n) \rightarrow \infty$}
\end{SNP}
\end{document}