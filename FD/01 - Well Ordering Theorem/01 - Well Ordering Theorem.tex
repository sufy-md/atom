\documentclass[a4paper, 14pt]{extarticle}

\usepackage{amsmath}
\usepackage{amssymb}
\usepackage{amsthm}
\usepackage{commath}
\usepackage[margin=0.5in]{geometry}
% \usepackage{lexend}
\usepackage{microtype}
\usepackage{parskip}
\usepackage{tikz}
\usepackage{tikz-cd}
\usepackage{tkz-euclide}
\usepackage{xparse}

\usetikzlibrary{calc,angles,quotes, positioning, shapes.geometric}

\theoremstyle{definition}
\newtheorem{dfn}{Definition}
\newtheorem{clm}{Claim}
\newtheorem{asn}{Assertion}
\newtheorem{thm}{Theorem}
\newtheorem{prb}{Problem}
\newtheorem{ans}{Answer}
\newtheorem{lm}{Lemma}
\newtheorem{rmk}{Remark}
\newtheorem{crl}{Corollary}
\newtheorem{ex}{Exercise}
\newtheorem{xmp}{Example}

\newcommand{\titleheader}[1]{\begin{centering}
\begin{LARGE}
\textbf{#1}
\end{LARGE}\\
\hrulefill

\vspace{-1.25\baselineskip}
\hrulefill
\end{centering}\\\\}

\def\changemargin#1#2{\list{}{\rightmargin#2\leftmargin#1}\item[]}
\let\endchangemargin=\endlist

\NewDocumentEnvironment{smrg}{}{\begin{changemargin}{0.5cm}{0.5cm}}{\end{changemargin}}

\NewDocumentEnvironment{SWP}{m m}
{%
  \vspace{-.9cm}%
  \begin{changemargin}{0.5cm}{0.5cm}%
  \noindent#1~#2
  \par
  \textbf{Proof.} 
}
{%
  \qed
  \end{changemargin}
}

\NewDocumentEnvironment{SNP}{m}
{%
  \vspace{-.9cm}%
  \begin{changemargin}{0.5cm}{0.5cm}%
  \noindent#1
}
{%
  \end{changemargin}
}

\newcommand{\bb}[1]{\mathbb{#1}}
\newcommand{\st}{\space \mid \space}
\newcommand{\union}[1]{\displaystyle\mathop{\cup}\limits_{#1}}
\newcommand{\intersect}[1]{\displaystyle\mathop{\cap}\limits_{#1}}
\newcommand{\paran}[1]{\left ( {#1} \right )}
\newcommand{\contra}{$\rightarrow\!\leftarrow$}
\newcommand{\kvec}[2]{({#1}_1 \dots {#1}_{#2})}
\newcommand{\pf}{\textbf{Proof.} }

\newcommand{\AnswerSection}{
    \newpage
    \section*{Answers to Exercises}
    \textit{The following are brief solutions or hints. You are encouraged to review the exercises before checking the answers.}
}
\begin{document}
\titleheader{Well Ordering Theorem}
Here we will have a look at an interesting property of the natural numbers. As an example, consider the sets $[0, 1], (1, 2)$, and $(-\infty, 0)$ in $\bb R$

The first set admits a minimum element -- $0$. The second set admits a \emph{greatest lower bound} -- $1$. Finally the third set is unbounded below.

Take any nonempty subset of the natural numbers, say $\{3n \st n \in \bb N\}$. This has a minimum element, $3$. Keep trying nonempty subsets until you build an intuition for the following statement.

\begin{SNP}{\thm}{Every nonempty subset of $\bb N$ has a minimum element}
\end{SNP}

We will present a proof in two parts. First, using induction, we will prove the property for finite nonempty subsets. Then we will generalise to all subsets.

\begin{SWP}{\lm}{Every nonempty finite subset of $\bb N$ has a least element}
We proceed by induction on the number of elements of a given finite $S \subset \bb N$. If $S$ has one element, i.e $S = \{m\}$, then $m$ is the least element of $S$.

If $S$ has $2$ elements, then by totality of $<$ one of them must be minimum in $S$.

Now suppose for some $k$, every subset $S$ of size $k$ has a least element, and let $S'$ be a subset of size $k + 1$. Indeed since $S'$ is nonempty, pick some $m' \in S'$ and consider $T := S' \setminus \{m'\}$. $T$ is therefore a subset of size $k$, and has a minimum element $m$ by induction hypothesis.

Now finally $\{m, m'\}$ is a subset of $\bb N$ of size $2$ and again admits a minimum. This is the required minimum of $S$.
\end{SWP}
Essentially, our argument boiled down to
$$
\min (\text{set of size } k + 1) = \min(\text{set of size } k \text{ obtained by removing some }m, m)
$$
Now we are in position to prove the infinite case.
\begin{smrg}
\pf (Of Theorem 1.) Let $S \subset \bb N$ be nonempty. If $S$ is finite we are done. Otherwise, take $k \in S$ and consider the set
\begin{align*}
S' &= \{n \in S \st n \leq k\} \\
&= S \cap \{n \in \bb N \st n \leq k\}
\end{align*}
i.e, those natural numbers which are smaller than or equal to $k$ and belong to $S$. $S'$ is nonempty because $k \in S'$. Further, $S'$ is finite. Thus $S'$ has a minimum element $m$. This is also the minimum element of $S$.\qed
\end{smrg}
\newpage
We can in fact go the other way. The principle of mathematical induction states that let $P(k)$ be a statement depending on the natural number $k$. If $P(0)$ is true and $P(k) \implies P(k + 1)$ for every $k$, then $P(n)$ is true for every $n$.

Suppose the Well Ordering Theorem is true. Suppose $P(0)$ is true, and $P(k) \implies P(k + 1)$. Can you show that $P(n)$ is true for every $n$?
\begin{SNP}{\ex}{
Show that Well Ordering Theorem implies Principle of Mathematical Induction (Hint. Use the above outline. Suppose that $P(k)$ is false for some $k > 0$, then the set $S$ of natural numbers for which $P$ is false is nonempty. Now what?)}
\end{SNP}

Indeed, Mathematical Induction and Well Ordering Theorem are \textbf{equivalent}. Neither of these can be proved without each other, and for discrete math we take these to be axioms.

\AnswerSection
\ans Let $S$ be the set $\{ k \in \bb N \st P(k) \text{ is false} \}$. Suppose for contradiction that $S$ is nonempty, so it admits a minimum $m$. We know $m \neq 0$ since we assumed $P(0)$ to be true. i.e, $m \geq 1$ or $m - 1 \geq 0$ and thus $m - 1$ is also natural. We know $m$ is the smallest $k$ for which $P(k)$ is false, therefore $P(m - 1)$ must be true. But we assumed $P(k) \implies P(k + 1)$ so $P(m)$ must also be true! $S$ must be an empty set.\qed
\end{document}