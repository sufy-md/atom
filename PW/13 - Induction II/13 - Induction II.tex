 \documentclass[a4paper, 14pt]{extarticle}

\usepackage{amsmath}
\usepackage{amssymb}
\usepackage{amsthm}
\usepackage{commath}
\usepackage[margin=0.5in]{geometry}
% \usepackage{lexend}
\usepackage{microtype}
\usepackage{parskip}
\usepackage{tikz}
\usepackage{tikz-cd}
\usepackage{tkz-euclide}
\usepackage{xparse}

\usetikzlibrary{calc,angles,quotes, positioning, shapes.geometric}

\theoremstyle{definition}
\newtheorem{dfn}{Definition}
\newtheorem{clm}{Claim}
\newtheorem{asn}{Assertion}
\newtheorem{thm}{Theorem}
\newtheorem{prb}{Problem}
\newtheorem{ans}{Answer}
\newtheorem{lm}{Lemma}
\newtheorem{rmk}{Remark}
\newtheorem{crl}{Corollary}
\newtheorem{ex}{Exercise}
\newtheorem{xmp}{Example}

\newcommand{\titleheader}[1]{\begin{centering}
\begin{LARGE}
\textbf{#1}
\end{LARGE}\\
\hrulefill

\vspace{-1.25\baselineskip}
\hrulefill
\end{centering}\\\\}

\def\changemargin#1#2{\list{}{\rightmargin#2\leftmargin#1}\item[]}
\let\endchangemargin=\endlist

\NewDocumentEnvironment{smrg}{}{\begin{changemargin}{0.5cm}{0.5cm}}{\end{changemargin}}

\NewDocumentEnvironment{SWP}{m m}
{%
  \vspace{-.9cm}%
  \begin{changemargin}{0.5cm}{0.5cm}%
  \noindent#1~#2
  \par
  \textbf{Proof.} 
}
{%
  \qed
  \end{changemargin}
}

\NewDocumentEnvironment{SNP}{m}
{%
  \vspace{-.9cm}%
  \begin{changemargin}{0.5cm}{0.5cm}%
  \noindent#1
}
{%
  \end{changemargin}
}

\newcommand{\bb}[1]{\mathbb{#1}}
\newcommand{\st}{\space \mid \space}
\newcommand{\union}[1]{\displaystyle\mathop{\cup}\limits_{#1}}
\newcommand{\intersect}[1]{\displaystyle\mathop{\cap}\limits_{#1}}
\newcommand{\paran}[1]{\left ( {#1} \right )}
\newcommand{\contra}{$\rightarrow\!\leftarrow$}
\newcommand{\kvec}[2]{({#1}_1 \dots {#1}_{#2})}
\newcommand{\pf}{\textbf{Proof.} }

\newcommand{\AnswerSection}{
    \newpage
    \section*{Answers to Exercises}
    \textit{The following are brief solutions or hints. You are encouraged to review the exercises before checking the answers.}
}
\begin{document}
\titleheader{Common Induction Mistakes}
Not so much `mistakes' as it is examples of the kind of mistake I pointed out at the end of the last discussion. Let's have a look.

\section*{All Horses are the Same Color}
Let $P(n)$ be the statement \texttt{In any set of $n$ horses, all horses are the same color.}

Then we make the following claim.
\begin{SWP}{\asn}{$P(n)$ is true for all $n \geq 1$.}(Base Case) Let $n = 1$. Then any set of one horse has all horses the same color, since there is only one horse. Thus $P(1)$ is trivially true.

(Induction Hypothesis) Assume $P(k)$ is true for some $k \geq 1$. That is, in any set of $k$ horses, all horses are the same color.

(Completion) Consider a set of $k + 1$ horses, $H_1, H_2, \ldots, H_{k+1}$.

By the induction hypothesis, the first $k$ horses, $H_1, H_2, \ldots, H_k$, are all the same color.

Again by the induction hypothesis, the last $k$ horses, $H_2, H_3, \ldots, H_{k+1}$, are all the same color.

Since $H_2$ is in both sets, it must be the case that all horses in the set of $k + 1$ horses are the same color.
\end{SWP}
Well? This argument seems to suggest that all horses are the same color, regardless of the number of horses. What went wrong?

My process relies on removing one horse from the set and considering the colors of the remaining horses.

However, this process \textit{fails} when proving $P(2)$ from $P(1)$ because when you remove just the one horse, there's nothing left.

\begin{footnotesize}{Quick sidenote -- all horses being the same color in an empty set is a vacuously true statement. We will look at those in the final discussion.}
\end{footnotesize}
\section*{Small Offset}
Let's have a look at one final example for induction.

Suppose you set out to show that $\sum\limits_{k=1}^{n} k = \dfrac{n^2+n+2}{2}$. If you checked that $f(n+1) = f(n) + (n+1)$, you would find that equality holds and make your conclusion.

Where you would go wrong, however, is checking the base case. You never light your string of crackers on fire, you just check that burning one will ignite the next.

\section*{Conclusion}
Light the fire. \textit{Check the base case}. That's all.

\end{document}