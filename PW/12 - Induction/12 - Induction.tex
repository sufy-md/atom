\documentclass[a4paper, 14pt]{extarticle}

\usepackage{amsmath}
\usepackage{amssymb}
\usepackage{amsthm}
\usepackage{commath}
\usepackage[margin=0.5in]{geometry}
% \usepackage{lexend}
\usepackage{microtype}
\usepackage{parskip}
\usepackage{tikz}
\usepackage{tikz-cd}
\usepackage{tkz-euclide}
\usepackage{xparse}

\usetikzlibrary{calc,angles,quotes, positioning, shapes.geometric}

\theoremstyle{definition}
\newtheorem{dfn}{Definition}
\newtheorem{clm}{Claim}
\newtheorem{asn}{Assertion}
\newtheorem{thm}{Theorem}
\newtheorem{prb}{Problem}
\newtheorem{ans}{Answer}
\newtheorem{lm}{Lemma}
\newtheorem{rmk}{Remark}
\newtheorem{crl}{Corollary}
\newtheorem{ex}{Exercise}
\newtheorem{xmp}{Example}

\newcommand{\titleheader}[1]{\begin{centering}
\begin{LARGE}
\textbf{#1}
\end{LARGE}\\
\hrulefill

\vspace{-1.25\baselineskip}
\hrulefill
\end{centering}\\\\}

\def\changemargin#1#2{\list{}{\rightmargin#2\leftmargin#1}\item[]}
\let\endchangemargin=\endlist

\NewDocumentEnvironment{smrg}{}{\begin{changemargin}{0.5cm}{0.5cm}}{\end{changemargin}}

\NewDocumentEnvironment{SWP}{m m}
{%
  \vspace{-.9cm}%
  \begin{changemargin}{0.5cm}{0.5cm}%
  \noindent#1~#2
  \par
  \textbf{Proof.} 
}
{%
  \qed
  \end{changemargin}
}

\NewDocumentEnvironment{SNP}{m}
{%
  \vspace{-.9cm}%
  \begin{changemargin}{0.5cm}{0.5cm}%
  \noindent#1
}
{%
  \end{changemargin}
}

\newcommand{\bb}[1]{\mathbb{#1}}
\newcommand{\st}{\space \mid \space}
\newcommand{\union}[1]{\displaystyle\mathop{\cup}\limits_{#1}}
\newcommand{\intersect}[1]{\displaystyle\mathop{\cap}\limits_{#1}}
\newcommand{\paran}[1]{\left ( {#1} \right )}
\newcommand{\contra}{$\rightarrow\!\leftarrow$}
\newcommand{\kvec}[2]{({#1}_1 \dots {#1}_{#2})}
\newcommand{\pf}{\textbf{Proof.} }

\newcommand{\AnswerSection}{
    \newpage
    \section*{Answers to Exercises}
    \textit{The following are brief solutions or hints. You are encouraged to review the exercises before checking the answers.}
}
\begin{document}
\titleheader{Lighting the Fuse}
The final proof technique I want to demonstrate is proof by induction. Induction is useful when you want to show some statement $P$ holds for all natural numbers $n \geq n_0$ for some fixed $n_0$.

Examples of these statements are the Fundamental Theorem of Arithmetic, $\sum _{k=1}^{n} k = \frac {n(n+1)}2$, the binomial theorem, $\cos nx$ is an $n$ degree polynomial in $\cos x$, the $\text{AM-GM}$ inequality etc.

The idea is to establish the truth for \textit{some} $n \in \mathbb N$ and use it to establish truth for other $n$ using recurrences, breaking larger numbers into smaller numbers, etc.

Now, you're familiar with induction. For completeness, I'll cover a basic example, point out some common pitfalls, and then get to some more convoluted examples. Let's start with a straightforward example:

\section*{Straightforward / Textbook Examples}
You're familiar with the following identity.
\begin{SNP}{\prb}Show that$$\sum\limits_{k=1}^{n} k = \dfrac {n(n+1)}2$$
\end{SNP}
Let's prove it by induction. We will show that the statement holds for all $n \geq 1$.
\begin{smrg}\textbf{Proof.} For $n = 1$, the statement clearly holds.\hfill{\footnotesize\itshape(This is Step 1. of an Induction Proof -- the base case.)}

Now, suppose the statement holds for some natural number $m \geq 1$.\hfill{\footnotesize\itshape(This is Step 2. of an Induction Proof -- the induction hypothesis.)}
\hfill{\footnotesize\itshape(We assume the statement holds for only one specific $m$ -- this is the ``weak'' induction hypothesis.)}
\end{smrg}
In Step $2$, we assume the statement to be true for \textit{some} subset of natural numbers, and try to prove it true for some other natural number. Here, I have assumed the statement is true for a natural number $m$, and I want to show it is true for $m + 1$.

Step $3$ is just to complete the proof, which I will do now.
\begin{smrg}
Consider the statement for $m + 1$. We have
\begin{align}
\sum\limits_{k=1}^{m+1} k &= \sum\limits_{k=1}^{m} k + (m + 1) \\
&= \dfrac {m(m+1)}2 + (m + 1)\\
&= \dfrac {m(m+1) + 2(m + 1)}2 \\
&= \dfrac {(m + 1)(m + 2)}2\\
&= \dfrac {(m + 1)((m + 1) + 1)}2
\end{align}
Where in transitioning from $(1)$ to $(2)$, we've used the induction hypothesis, that the statement is true for $m$.\qed
\end{smrg}
This is an example of using a recurrence ($P(m) \implies P(m + 1)$) alongside a base case ($P(1)$) to complete the proof.

Now let's have a look at strong induction.
\begin{SWP}{\prb}{Show that every positive integer $n > 1$ is either prime or a product of primes.}(Base Case) $n = 2$ is prime.

(Induction Hypothesis) Assume the statement holds for all integers $k$ such that $2 \leq k \leq m$ for some $m \geq 2$.\hfill{\footnotesize\itshape(We assume the statement holds for all integers up to $m$ -- this is the ``strong'' induction hypothesis.)}

(Complete the Proof) Consider $m + 1$. If $m + 1$ is prime, we are done. Otherwise, it can be expressed as a product of two integers $a$ and $b$, where $2 \leq a, b < m + 1$. By the induction hypothesis, both $a$ and $b$ can be expressed as products of primes. Therefore, $m + 1$ can also be expressed as a product of primes.
\end{SWP}
This is an example of the ``building block'' approach. I did not exploit the truth of the statement at $m$. I just used some arbitrary numbers smaller than $m+1$ (not even knowing which ones) as ``building blocks'' for my proof.

\section*{Some Exotic Examples}
\subsection*{Euclid's Lemma for $n = 2$}
Recall we used Euclid's Lemma when proving an integer is even iff its square is even.

Here, we will prove the lemma for $n = 2$, i.e. given an integer $m$, there is a unique $n$ such that precisely one of $m = 2n$ or $m = 2n + 1$ holds.
\newpage
We will only concern ourselves with existence. Uniqueness and showing exactly one holds is not relevant to induction.
\begin{SWP}{\prb}{Show that for every integer $m$, there exists a unique integer $n$ such that either $m = 2n$ or $m = 2n + 1$.}We will take three subcases -- $m < 0$, $m = 0$, and $m > 0$.

Subcase $m = 0$ is trivial, as $n = 0$ satisfies the condition.

Consider the subcase $m > 0$.
(Base Cases) For $m = 1$, we have $1 = 2\times 0 + 1$ and for $m = 2$, we have $2 = 2 \times 1$. So the base cases hold.

(Induction Hypothesis) Suppose for some positive integer $k$, Euclid's Lemma holds.

(Completion) Consider $k + 1$. If $k$ has the form $2n$, then $k + 1 = 2n + 1$.

If $k$ has the form $2n + 1$, then $k + 1 = 2(n + 1)$.

Thus Euclid's Lemma holds for $k + 1$.\qed

Finally consider the subcase $m < 0$.

(Base Cases) For $m = -1$, we have $-1 = 2\times (-1) + 1$ and for $m = -2$, we have $-2 = 2 \times (-1)$. So the base cases hold.

(Induction Hypothesis) Suppose for some negative integer $k$, Euclid's Lemma holds.

(Completion) Consider $k - 1$. If $k$ has the form $2n$, then $k - 1 = 2(n - 1) + 1$.

If $k$ has the form $2n + 1$, then $k - 1 = 2n$.

Thus Euclid's Lemma holds for $k - 1$.
\end{SWP}
And that does it. It is impossible for us to have left out any integer (think about why this is true). I list this as an exotic example because of the subcase division and the two base cases.

\subsection*{AM-GM Inequality}
Let me state the problem first.
\begin{SNP}{\thm}
For any non-negative real numbers $x_1, x_2, \ldots, x_n$, we have
\[
\dfrac{x_1 + x_2 + \cdots + x_n}{n} \geq \sqrt[n]{x_1 x_2 \cdots x_n}
\]
with equality if and only if $x_1 = x_2 = \cdots = x_n$.
\end{SNP}
Again, we will concern ourselves with just the inequality. The condition for equality is not relevant to induction.

We will perform an induction on \textit{the number of variables}, and the way we will ensure every $n > 0$ is hit is unlike anything you will ever encounter.
\begin{smrg}\textbf{Proof.} For $n = 1$, the inequality is trivially true, as both sides are equal to $x_1$.

(Induction Hypothesis) Suppose that given \textit{any} $k$ positive reals with $1 \leq k \leq m$, the inequality holds.

(Completion) First we will show the inequality holds for any $2m$ positive reals. Let $x_1, x_2, \ldots, x_{2m}$ be any $2m$ positive reals.

Group them into $x_1, x_2 \ldots x_m$ and $x_{m+1}, x_{m+2} \ldots x_{2m}$.

By Induction Hypothesis, we can apply the inequality to both groups:
\begin{align}
\dfrac{x_1 + x_2 + \cdots + x_m}{m} &\geq \sqrt[m]{x_1 x_2 \cdots x_m} \\
\dfrac{x_{m+1} + x_{m+2} + \cdots + x_{2m}}{m} &\geq \sqrt[m]{x_{m+1} x_{m+2} \cdots x_{2m}}
\end{align}

Note the LHS for both inequalities. They're both positive real numbers. First we observe that the AM of the LHSs is $\ge$ the AM of the RHSs.
\begin{equation}
\dfrac{\dfrac{x_1 + x_2 + \ldots + x_m}{m} + \dfrac{x_{m+1} + \ldots + x_{2m}}{m}}{2} \geq \dfrac{\sqrt[m]{x_1 x_2 \ldots x_m} + \sqrt[m]{x_{m+1} \ldots x_{2m}}}{2}
\end{equation}

Applying AM-GM on RHS for $2$ variables (by IH) we get
\begin{align}
\dfrac{\sqrt[m]{x_1 x_2 \ldots x_m} + \sqrt[m]{x_{m+1} \ldots x_{2m}}}{2} &\geq \sqrt{\sqrt[m]{x_1 x_2 \ldots x_m}\cdot\sqrt[m]{x_{m+1} \ldots x_{2m}}}\\
&= \sqrt[2m]{x_1x_2\ldots x_{2m}}
\end{align}

Combining $(8)$, $(9)$, and $(10)$ we get the AM-GM inequality for $2m$ variables.
\[
\dfrac{x_1 + x_2 + \cdots + x_{2m}}{2m} \geq \sqrt[2m]{x_1 x_2 \cdots x_{2m}}
\]

Finally, we'll show the inequality holds for $m - 1$ positive reals with the same IH.

Let $x_1, x_2, \ldots, x_{m-1}$ be any $m - 1$ positive reals. Define $x_m$ as
\[
x_m = \dfrac{x_1 + x_2 + \cdots + x_{m-1}}{m - 1}
\]

And apply the inequality to $x_1, x_2, \ldots, x_{m-1}, x_m$.
\[
\dfrac{x_1 + \dots + x_{m-1} + \frac{x_1 + \dots + x_{m-1}}{m-1}}{m} \geq \sqrt[m]{x_1 \dots x_{m-1} \cdot \left(\frac{x_1 + \dots + x_{m-1}}{m-1}\right)}
\]

The LHS simplifies to
\[
\dfrac{x_1 + \dots + x_{m-1}}{m - 1}
\]

Thus
\[
\left(\dfrac{x_1 + \dots + x_{m-1}}{m-1}\right)^m \geq x_1\dots x_{m-1}\cdot\left(\frac{x_1+\dots+x_{m-1}}{m-1}\right)
\]

Removing one copy of the fraction from either side,
\[
\left(\dfrac{x_1 + \dots + x_{m-1}}{m-1}\right)^{m-1} \geq x_1\dots x_{m-1}
\]
And we're done.\qed
\end{smrg}
Or are we? Those of you still awake would have noticed our proof for $k \rightarrow 2k$ relies on the inequality being true for $2$ variables. But the inequality being true for $2$ variables relies on the $k \rightarrow 2k$ implication holding in the first place.

This is a common pitfall in induction proofs. We need to find a different proof for $2$ variables. That is easily done by considering $(\sqrt {a_1} - \sqrt{a_2})^2 \geq 0$.

What I wanted to emphasise is that you have to have a \textit{very} clear image of how your chain of implications works, and fix it wherever necessary. More on this in the next.
\end{document}