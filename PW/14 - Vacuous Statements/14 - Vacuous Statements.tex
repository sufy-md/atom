\documentclass[a4paper, 14pt]{extarticle}

\usepackage{amsmath}
\usepackage{amssymb}
\usepackage{amsthm}
\usepackage{commath}
\usepackage[margin=0.5in]{geometry}
% \usepackage{lexend}
\usepackage{microtype}
\usepackage{parskip}
\usepackage{tikz}
\usepackage{tikz-cd}
\usepackage{tkz-euclide}
\usepackage{xparse}

\usetikzlibrary{calc,angles,quotes, positioning, shapes.geometric}

\theoremstyle{definition}
\newtheorem{dfn}{Definition}
\newtheorem{clm}{Claim}
\newtheorem{asn}{Assertion}
\newtheorem{thm}{Theorem}
\newtheorem{prb}{Problem}
\newtheorem{ans}{Answer}
\newtheorem{lm}{Lemma}
\newtheorem{rmk}{Remark}
\newtheorem{crl}{Corollary}
\newtheorem{ex}{Exercise}
\newtheorem{xmp}{Example}

\newcommand{\titleheader}[1]{\begin{centering}
\begin{LARGE}
\textbf{#1}
\end{LARGE}\\
\hrulefill

\vspace{-1.25\baselineskip}
\hrulefill
\end{centering}\\\\}

\def\changemargin#1#2{\list{}{\rightmargin#2\leftmargin#1}\item[]}
\let\endchangemargin=\endlist

\NewDocumentEnvironment{smrg}{}{\begin{changemargin}{0.5cm}{0.5cm}}{\end{changemargin}}

\NewDocumentEnvironment{SWP}{m m}
{%
  \vspace{-.9cm}%
  \begin{changemargin}{0.5cm}{0.5cm}%
  \noindent#1~#2
  \par
  \textbf{Proof.} 
}
{%
  \qed
  \end{changemargin}
}

\NewDocumentEnvironment{SNP}{m}
{%
  \vspace{-.9cm}%
  \begin{changemargin}{0.5cm}{0.5cm}%
  \noindent#1
}
{%
  \end{changemargin}
}

\newcommand{\bb}[1]{\mathbb{#1}}
\newcommand{\st}{\space \mid \space}
\newcommand{\union}[1]{\displaystyle\mathop{\cup}\limits_{#1}}
\newcommand{\intersect}[1]{\displaystyle\mathop{\cap}\limits_{#1}}
\newcommand{\paran}[1]{\left ( {#1} \right )}
\newcommand{\contra}{$\rightarrow\!\leftarrow$}
\newcommand{\kvec}[2]{({#1}_1 \dots {#1}_{#2})}
\newcommand{\pf}{\textbf{Proof.} }

\newcommand{\AnswerSection}{
    \newpage
    \section*{Answers to Exercises}
    \textit{The following are brief solutions or hints. You are encouraged to review the exercises before checking the answers.}
}
\begin{document}
\titleheader{Empty Promises}
I would like to dedicate this final discussion to vacuous statements - statements that are true \textit{by default} - or by lack of counterexamples.

Let me show you.

Suppose you have an empty bucket. Consider the statement \texttt{all balls in this bucket are blue}. Is this statement true or false?

Well, can you produce a ball from the bucket that \textit{isn't} blue? No, of course not. Thus the statement must be true.

Now you could argue that one cannot produce a blue ball either. But get this, my statement was \texttt{if a ball is in this bucket, it is blue}. You \textit{cannot} disprove it, because doing so would require you to produce a counterexample from the bucket. And as we know, the bucket is empty. Hence there are no counterexamples.

\section*{A Little bit More}
Vacuous statements have the form $\forall \space r, P(r) \implies Q(r)$. The difference is that $r$ either comes from an empty collection, or for every $r$, $\neg P(r)$ holds. (Recall the equivalent of an implication).

Have some more examples.
\begin{enumerate}
    \item Every even injective function from $\mathbb R$ to itself is continuous
    \item Every skew-symmetric nonzero $1\times 1$ matrix is invertible.
    \item Every even integer of the form $24k+13$ is a perfect square
\end{enumerate}
and so on. This is important because sometimes, it could be the case that a forall statement is vacuously true, but you use it the wrong way.

I really don't have any examples for this other than the footnote in PW-13.

\section*{Conclusion}
That concludes my little course on proofwriting and logic. I hope it was useful to you, and that you learned something. 

If you encounter questions, typos, or anything else, please feel free to reach out. Cya.
\end{document}