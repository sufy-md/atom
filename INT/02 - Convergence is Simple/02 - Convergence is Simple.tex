\documentclass[a4paper, 14pt]{extarticle}

\usepackage{amsmath}
\usepackage{amssymb}
\usepackage{amsthm}
\usepackage{commath}
\usepackage[margin=0.5in]{geometry}
% \usepackage{lexend}
\usepackage{microtype}
\usepackage{parskip}
\usepackage{tikz}
\usepackage{tikz-cd}
\usepackage{tkz-euclide}
\usepackage{xparse}

\usetikzlibrary{calc,angles,quotes, positioning, shapes.geometric}

\theoremstyle{definition}
\newtheorem{dfn}{Definition}
\newtheorem{clm}{Claim}
\newtheorem{asn}{Assertion}
\newtheorem{thm}{Theorem}
\newtheorem{prb}{Problem}
\newtheorem{ans}{Answer}
\newtheorem{lm}{Lemma}
\newtheorem{rmk}{Remark}
\newtheorem{crl}{Corollary}
\newtheorem{ex}{Exercise}
\newtheorem{xmp}{Example}

\newcommand{\titleheader}[1]{\begin{centering}
\begin{LARGE}
\textbf{#1}
\end{LARGE}\\
\hrulefill

\vspace{-1.25\baselineskip}
\hrulefill
\end{centering}\\\\}

\def\changemargin#1#2{\list{}{\rightmargin#2\leftmargin#1}\item[]}
\let\endchangemargin=\endlist

\NewDocumentEnvironment{smrg}{}{\begin{changemargin}{0.5cm}{0.5cm}}{\end{changemargin}}

\NewDocumentEnvironment{SWP}{m m}
{%
  \vspace{-.9cm}%
  \begin{changemargin}{0.5cm}{0.5cm}%
  \noindent#1~#2
  \par
  \textbf{Proof.} 
}
{%
  \qed
  \end{changemargin}
}

\NewDocumentEnvironment{SNP}{m}
{%
  \vspace{-.9cm}%
  \begin{changemargin}{0.5cm}{0.5cm}%
  \noindent#1
}
{%
  \end{changemargin}
}

\newcommand{\bb}[1]{\mathbb{#1}}
\newcommand{\st}{\space \mid \space}
\newcommand{\union}[1]{\displaystyle\mathop{\cup}\limits_{#1}}
\newcommand{\intersect}[1]{\displaystyle\mathop{\cap}\limits_{#1}}
\newcommand{\paran}[1]{\left ( {#1} \right )}
\newcommand{\contra}{$\rightarrow\!\leftarrow$}
\newcommand{\kvec}[2]{({#1}_1 \dots {#1}_{#2})}
\newcommand{\pf}{\textbf{Proof.} }

\newcommand{\AnswerSection}{
    \newpage
    \section*{Answers to Exercises}
    \textit{The following are brief solutions or hints. You are encouraged to review the exercises before checking the answers.}
}
\begin{document}
\titleheader{Convergence is Simple}
Consider the following limit
$$
L := \lim_{n \rightarrow \infty} 1 - \dfrac{1}{n}
$$
Indeed, $L = 1$. Here, we say that \emph{the sequence} $1 - \frac{1}{n}$ \emph{converges to} $1$. Intuitively, we know what convergence of a sequence should mean, but how do we define it?

We want to quantify a simple idea. If we say $a_n \rightarrow L$ as $n \rightarrow \infty$ it should be if and only if $a_n$ gets closer and closer to $L$ as $n \rightarrow \infty$.

That in and of itself isn't enough. For starters, the sequence $1 - \frac 1 n$ also gets closer and closer to, say, $1.5$ as $n \rightarrow \infty$. If you further try some shenanigans along the lines of ``as $n \rightarrow \infty$, $a_n$ should \emph{go towards} $L$'', that doesn't work either, because the sequence $\frac{(-1)^n}{n}$ doesn't go anywhere, it alternates and eventually diminishes.

Let's sit down for a conversation, shall we? You claim that the sequence $1 - \frac 1 n$ converges to $1$. I ask you what that means. You say ``well, the sequence gets closer and closer to $1$ as $n \rightarrow \infty$''. I ask you how close to $1$ does the sequence get, really?

Now you do some thinking. How close? Eventually, you answer - ``as close as you want''. I'm slightly taken aback, but I decide to test your claim. Does the sequence ever land within $0.5$ of $1$? Sure, just take $n > 2$. Within $0.33$? Sure, take $n > 3$. Within $0.25$? Sure, take $n > 4$, and so on.

Eventually I get frustrated and give you some arbitrary number $\epsilon$ and ask you - does your sequence get closer than $\epsilon$ to $1$?

\begin{smrg}
\begin{center}
\begin{tikzpicture}[x=4cm]
  % Draw number line
  \draw[->] (-1,0) -- (3,0) node[right] {$x$};

  % Mark point at 1
  \draw (1,0) node[below=4pt] {$1$} -- (1,0.1);

  % Define epsilon
  \def\eps{0.5}

  % Open interval around 1
  \draw[blue, thick] (1 - \eps, 0) -- (1 + \eps, 0);
  \filldraw[fill=white, draw=black, thick] (1 - \eps, 0) circle (2pt);
  \filldraw[fill=white, draw=black, thick] (1 + \eps, 0) circle (2pt);

  % Optional labels
  \node[below=4pt] at (1 - \eps, 0) {$1 - \epsilon$};
  \node[below=4pt] at (1 + \eps, 0) {$1 + \epsilon$};
\end{tikzpicture}
Does the sequence fall within $\epsilon$ of $1$?
\end{center}
\end{smrg}

Now you do some thinking. You want $1 - a_n < \epsilon$, or $\frac 1 n < \epsilon$. Can you guarantee this for a large enough $n$? Yes! We use the Archimedean Property of $\bb R$ to guarantee that there is some natural number $N$ such that
$$
N > \dfrac 1 \epsilon
$$
and therefore for every $n \geq N$, we have
$$
\epsilon > \dfrac 1 n
$$
Thus whenever $n > \frac{1}{\epsilon}$, the term $a_n$ falls within $\epsilon$ distance of $1$.

A subtle but important feature of this formulation is this -- let $\epsilon$ be some fixed number. Then, there is some fixed $N$ such that ALL of $a_N, a_{N + 1}, a_{N + 2} \dots$ fall within $\epsilon$ of $1$

It is necessary for us to impose this condition that all values beyond a fixed $N$ fall within the stipulated distance of the limit. This and only this captures in the truest sense what it means for a sequence to get closer and closer to a number.

For a nonexample, consider the sequence
$$
-1, 1, -1, 1 \dots
$$
where the general term is given by $(-1)^n$. Here, \emph{infinitely} many terms of the sequence are arbitrarily close to $1$ (all terms of the form $a_{2n}$ are equal to $1$ and hence their distance from $n$ is $0$, which is smaller than $\epsilon$ for every positive $\epsilon$). However, would you say that $a_n \rightarrow 1$? Probably not. Does our intuition capture this?

Remember, we want that given some $\epsilon > 0$, there should be some $N$ such that $a_{N + 1}, a_{N + 2}, a_{N + 3} \dots$ all fall within $\epsilon$ of the limit. If $1$ was the limit and we take $\epsilon = 0.5$, then the distance between $1$ and either $a_{N + 1}$ or $a_{N + 2}$ wil be $2$ (one of them is guaranteed to be $-1$). Since $\epsilon = 0.5$, the sequence fails to fall within $\epsilon$ of $1$, therefore it doesn't converge to $1$.

We are ready to formalise the definition
\begin{SNP}{\dfn}
{
	Say a sequence $a_n$ \emph{converges to} $L \in \bb R$ if and only if the following holds:

	Given some $\epsilon > 0$, there is some $N$ such that ignoring the first $N$ terms of the sequence, all other terms lie within $\epsilon$ distance of $L$.

	i.e., for every $\epsilon > 0$, there is some $N \in \bb N$, such that $\abs{L - a_n} < \epsilon$ for every $n = N + 1, N + 2, N + 3 \dots$
}
\end{SNP}
\end{document}