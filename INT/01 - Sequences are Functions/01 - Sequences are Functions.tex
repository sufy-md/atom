\documentclass[a4paper, 14pt]{extarticle}

\usepackage{amsmath}
\usepackage{amssymb}
\usepackage{amsthm}
\usepackage{commath}
\usepackage[margin=0.5in]{geometry}
% \usepackage{lexend}
\usepackage{microtype}
\usepackage{parskip}
\usepackage{tikz}
\usepackage{tikz-cd}
\usepackage{tkz-euclide}
\usepackage{xparse}

\usetikzlibrary{calc,angles,quotes, positioning, shapes.geometric}

\theoremstyle{definition}
\newtheorem{dfn}{Definition}
\newtheorem{clm}{Claim}
\newtheorem{asn}{Assertion}
\newtheorem{thm}{Theorem}
\newtheorem{prb}{Problem}
\newtheorem{ans}{Answer}
\newtheorem{lm}{Lemma}
\newtheorem{rmk}{Remark}
\newtheorem{crl}{Corollary}
\newtheorem{ex}{Exercise}
\newtheorem{xmp}{Example}

\newcommand{\titleheader}[1]{\begin{centering}
\begin{LARGE}
\textbf{#1}
\end{LARGE}\\
\hrulefill

\vspace{-1.25\baselineskip}
\hrulefill
\end{centering}\\\\}

\def\changemargin#1#2{\list{}{\rightmargin#2\leftmargin#1}\item[]}
\let\endchangemargin=\endlist

\NewDocumentEnvironment{smrg}{}{\begin{changemargin}{0.5cm}{0.5cm}}{\end{changemargin}}

\NewDocumentEnvironment{SWP}{m m}
{%
  \vspace{-.9cm}%
  \begin{changemargin}{0.5cm}{0.5cm}%
  \noindent#1~#2
  \par
  \textbf{Proof.} 
}
{%
  \qed
  \end{changemargin}
}

\NewDocumentEnvironment{SNP}{m}
{%
  \vspace{-.9cm}%
  \begin{changemargin}{0.5cm}{0.5cm}%
  \noindent#1
}
{%
  \end{changemargin}
}

\newcommand{\bb}[1]{\mathbb{#1}}
\newcommand{\st}{\space \mid \space}
\newcommand{\union}[1]{\displaystyle\mathop{\cup}\limits_{#1}}
\newcommand{\intersect}[1]{\displaystyle\mathop{\cap}\limits_{#1}}
\newcommand{\paran}[1]{\left ( {#1} \right )}
\newcommand{\contra}{$\rightarrow\!\leftarrow$}
\newcommand{\kvec}[2]{({#1}_1 \dots {#1}_{#2})}
\newcommand{\pf}{\textbf{Proof.} }

\newcommand{\AnswerSection}{
    \newpage
    \section*{Answers to Exercises}
    \textit{The following are brief solutions or hints. You are encouraged to review the exercises before checking the answers.}
}
\begin{document}
\titleheader{Sequences are Functions}
A \emph{sequence}, essentially, is a list of things -- numbers, points, sets, could be anything.

\begin{itemize}
	\item An arithmetic progression is a sequence, where the $n-$th term is given by $a + (n - 1)d$. This is a countably infinite list of numbers
	\item Consider the polynomial $x^3 - x$. This has zeros at $-1, 0, 1$. This is a finite list of numbers.
	\item Consider the lines in the Cartesian plane of slope $4$, the general given by $y = 4x + c$ where $c$ is a real number. This is an \emph{uncountable} list of sets (since lines are sets).
	\item Suppose there is a set, $\mathcal F$, that contains functions from a set $A$ to another set $B$. Fix some $a \in A$. Then we can list out the values $a$ takes across different functions in $\mathcal F$, and refer to the value $f(a)$ by $a_f$ instead. This is a list of elements of $B$.
\end{itemize}

You should notice, that each list, i.e., each sequence, is written \emph{with respect to} elements of some other set.

\begin{itemize}
	\item The terms of an arithmetic progression are defined by a parameter $n \in \bb N$ - so they're listed with respect to $\bb N$. i.e., each element of the progression is \emph{referred to} by an element of $\bb N$
	\item A little subtle, but if we say that the roots of the polynomial are $\lambda_1, \lambda_2,$ and $\lambda_3$; we implicitly list the roots with respect to the set $\{1, 2, 3\}$. i.e., each of the roots is \emph{referred to} by an element of the set.
	\item This should be the clearest one. Each line $y = 4x + c$ is \emph{referred to} by a real number $c$, and thus the sequence is with respect to $\bb R$
	\item Each value is \emph{referred to} by some function $f \in \mathcal F$, and thus the sequence is with respect to $\mathcal F$
\end{itemize}

If a sequence is written with respect to a set $\mathcal I$, we usually say it is \emph{indexed by} $\mathcal I$. For example, consider an AP with $a = 2$ and $d = 3$.

Then the set
$$
\{2, 5, 8, 11 \dots\}
$$
is indexed by $\bb N$ and
$$a_n = 2 + 3(n - 1) = 3n -1$$
Observe -- when we index a set $S$ with respect to another set $\mathcal I$ -- then for each $\iota \in \mathcal I$ there must be some element $s_\iota \in S$. Think of the indexing set as an actual index. For each chapter listed in the index of a book, a chapter exists in the actual content. Similarly, for each element in the index set, an element exists in the set being indexed. So for example, if a set is indexed by the natural numbers, it makes sense to say ``the $n$th element'' of that set. Does an $n-$th real number make sense? No. Does the $n-$th term of an AP make sense? Yes.

Thus, intuitively, every sequence is a list with respect to some index set. We've established that if we index some set $S$ with respect to $\mathcal I$, then ``the $\iota$-th element of $S$'' with $\iota \in \mathcal I$ should make sense.

Where have we seen this before? I have a set ($\mathcal I$), and for each element of the set ($\iota$), I want to assign it an element ($s_\iota$) of another set ($S$).

Indeed, a \emph{function} is something that does exactly this!

Therefore, for a set $S$ and an index set $\mathcal I$, it makes sense to define a sequence as
\begin{SNP}{\dfn
}{An $S$-valued sequence indexed by $\mathcal I$ is a function $f: \mathcal I \rightarrow S$
}
\end{SNP}

And indeed, the informal $s_\iota$ can be replaced by the formal $f(\iota)$, although the former can still be used as shorthand. (In fact, you list arithmetic progressions using that shorthand! $\kvec{a}{n}$ etc.)
\end{document}